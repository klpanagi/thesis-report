\chapter{Μελλοντικές επεκτάσεις}
\label{chapter:future_work}

Είναι σημαντικό να ερευνηθεί η εκτέλεση των πειραμάτων και στο ενσωματωμένο σύστημα
Jetson TΧ1. Το Jetson TX1 είναι το τελευταίο ενσωματωμένο σύστημα της Nvidia,
με αρχιτεκτονική επεξεργαστή 64bit και μονάδα GPU αρχιτεκτονικής Maxwell με
256 πυρήνες, το οποίο έχει σχεδιαστεί για χρήση σε εφαρμογές βαθιάς
μηχανικής μάθησης. Η αρχιτεκτονική Maxwell της μονάδας GPU είναι συμβατή
με την 5η έκδοση της βιβλιοθήκης cuDNN, η οποία ενσωματώνει ρουτίνες
εκτέλεσης των διαφόρων επιπέδων ενός CNN κατευθείαν πάνω στην GPU.

Επίσης, θα μπορούσαν να εκτελεστούν τα πειράματα με χρήση της βιβλιοθήκης
Tensorflow και άρα να ληφθούν αποτελέσματα τα οποία αφορούν την απόδοση
σε χρόνο εκτέλεσης μεταξύ των βιβλιοθηκών Theano και Tensorflow.
Η επιλογή χρήσης μίας εκ των 2 αυτών βιβλιοθηκών στο Keras ορίζεται μέσα από
ένα αρχείο ρύθμισης. Δηλαδή δεν απαιτούνται αλλαγές στον ήδη υπάρχον πηγαίο κώδικα
των υλοποιήσεων που έγιναν κατά την διάρκεια της παρούσας διπλωματική εργασίας.

Επιπρόσθετα θα μπορούσαν να γίνουν μετρήσεις της κατανάλωσης ισχύος στον
επεξεργαστή Intel i7-6700 και να συγκριθούν με τις αντίστοιχες
που παρουσιάστηκαν για το ενσωματωμένο σύστημα Jetson TK1. Οι
μετρήσεις αυτές θα μπορούσαν να χρησιμοποιηθούν στην συνέχεια για σύγκριση
της απόδοσης (σε χρόνο εκτέλεσης) των 2 συστημάτων σε σχέση με την κατανάλωση
ισχύος (performance per watt - GFlops/Watt).

Θα είχε ενδιαφέρον να αναπτυχθούν οι συγκεκριμένες υλοποιήσεις στο \emph{Cloud},
να ληφθούν μετρήσεις που αφορούν τον συνολικό χρόνο εκτέλεσης κλήσης
της υπηρεσίας αυτής και να συγκριθούν με τους χρόνους εκτέλεσης στο Jetson TK1
που παρουσιάστηκαν στο \autoref{chapter:experiments}.


Τέλος, αξίζει να ερευνηθεί το κομμάτι της εκπαίδευσης των CNN
με χρήση του εργαλείου \emph{DIGITS}\footnote{Nvidia DIGITS: \url{https://devblogs.nvidia.com/parallelforall/deep-learning-object-detection-digits/}} και η περεταίρω ανάπτυξη στοχευμένων εφαρμογών ρομποτικής
(στο ROS\footnote{\url{http://www.ros.org/}}).

