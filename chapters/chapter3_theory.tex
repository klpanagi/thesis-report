\chapter{Τεχνικές Βαθιάς Μηχανικής Μάθησης και Αναγνώριση Αντικειμένωνν στον χώρο}
\label{chapter:theory}

%Στο κεφάλαιο αυτό, παρουσιάζεται και αναλύεται το θεωρητικό υπόβαθρο,
%στο οποίο στηρίχτηκαν οι υλοποιήσεις, στην παρούσα διπλωματική εργασία.

%Στο Κεφάλαιο \ref{sec:theory_ml}, παρουσιάζονται οι βασικές αρχές
%και τεχνικες \emph{Μηχανικής Μάθησης} (Machine Learning - ML).

Τα τελευταία χρόνια, ο κλάδος της Τεχνιτής Νοημοσύνης είναι ένας από τους πιο ραγδαία
αναπτυσσόμενους κλάδους της επιστήμης της πληροφορικής με τεράστιο
ερευνητικό και πρακτικό ενδιαφέρον; διαιρείται σε δύο υπο\_\_\_; την \emph{συμβολική τεχνιτή
νοημοσύνη} και την \emph{υποσυμβολική τεχνητή νοημοσύνη}. Η πρώτη προσπαθεί να
επιλύσει τα προβλήματα χρησιμοποιώντας αλγοριθμικές διαδικασίες, δηλαδή
σύμβολα και λογικούς κανόνες, ενώ η δεύτερη προσπαθεί να αναπαράγει την
ανθρώπινη "ευφυία" μέσα από την χρήση αριθμητικών μοντέλων
που με την σύνθεσή τους προσομοιώνουν την λειτουργία του ανθρώπινου εγκεφάλου
(υπολογιστική νοημοσύνη).
Η ικανότητα ενός νοούμενου (AI) συστήματος, να αποκτά από μόνο του γνώση,
εξάγοντας πρότυπα ή/και χαρακτηριστικά σημεία από τα δεδομένα,
είναι γνωστή ώς \emph{Μηχανική Μάθηση} (ML).
Η εισαγωγή του κλάδου του ML στην επιστήμη των υπολογιστών,
επέτρεψε στους υπολογιστές να μπορούν να αντιμετωπίσουν προβλήματα που εμπλέκουν
την αντίληψη για τον πραγματικό κόσμο και να πέρνουν υποκειμενικές αποφάσεις.
\begin{figure}[H]
  \centering
  \includegraphics[width=0.9\textwidth]{./images/chapter3/AI_1.jpg}
  % \caption[Τεχνητή Νοημοσύνη]{Τεχνητή Νοημοσύνη}
  \label{fig:AI_1}
\end{figure}
Η χρησιμοποίηση αλγορίθμων ML, επιτρέπουν σε συστήματα AI
να προσαρμόζονται εύκολα σε καινούργια έργα, με ελάχιστη επέμβαση από τον άνθρωπο.
Για παράδειγμα, ένα Νευρωνικό Δίκτυο που έχει εκπεδευτεί να αναγνωρίζει γάτους σε εικόνες,
δεν απαιτεί να σχεδιαστεί και να εκπαιδευτεί από το μηδέν για να έχει την ικανότητα
να αναγνωρίζει και σκύλους.

Πολλά προβλήματα, ενώ μέχρι και πριν μία μερικά χρόνια λύνονταν με
"χειρόγραφη", προγραμματισμένη από τον άνθρωπο γνώση, σήμερα χρησιμοποιούνται
αλγόριθμοι ML για την επίλυσή τους. Πιο κάτω παρατίθενται μερικά
παραδείγματα:
\begin{itemize}
  \item{Αναγνώριση ομιλίας - Speech Recognition}
  \item{Μηχανική όραση - Computer Vision}
  \begin{itemize}
    \item{Αναγνώριση αντικειμένων σε εικόνες - Object Recognition}
    \item{Αναγνώριση και εντοπισμός της θέσης αντικειμένων σε εικόνες - Object Detection}
  \end{itemize}
  \item{Αναγνώριση ηλεκτρονικών επιθέσεων στο διαδίκτυο - Cyberattack detection}
  \item{Επεξεργασία φυσικής γλώσσας - Natural Language Processing}
  \begin{itemize}
    \item{Κατανόηση της φυσικής γλώσσας του ανθρώπου - Natural Language Understanding}
    \item{Μοντελοποίηση και χρησιμοποίηση της φυσικής γλώσσας του ανθρώπου από μηχανές - Natural Language Generation}
  \end{itemize}
  \item{Μηχανές αναζήτησης}
\end{itemize}

Τα προβλήματα Μηχανικής Μάθησης χωρίζονται σε τρεις μεγάλες κατηγορίες:
\begin{itemize}
  \item{Υπό επίβλεψη Μάθηση - Supervised Learning:
      Στο υπολογιστικό σύστημα δίνονται παραδείγματα εισόδου και επιθυμητής εξόδου,
      δηλαδή στα δεδομένα έχουν προηγουμένως ανατεθεί ετικέτες(labels),
      και στόχος είναι να μάθει ένα γενικό κανόνα αντιστοίχισης της εισόδου στην επιθυμητή έξοδο.
      Η αναγνώριση αντικειμένων σε εικόνες είναι ένα πρόβλημα που ανήκει σε αυτή την κατηγορία.
    }
  \item{Χωρίς επίβλεψη Μάθηση - Unsupervised Learning:
      Τα δεδομένα δεν έχουν ετικέτες (labels), αφήνοντας έτσι τον αλγόριθμο ML να βρεί
      από μόνος του δομές στα δεδομένα εισόδου.
    }
  \item{Εκμάθηση δια ανταμοιβής - Reinforcement Learning:
      Ο πράκτορας αλληλεπιδρά με ένα δυναμικό περιβάλλον στο οποίο πρέπει να
      εκτελέσει ένα συγκεκριμένο στόχο, χωρίς την ύπαρξη ενός "δασκάλου" που να
      ορίζει ρητά αν έχει φθάσει κοντά στον στόχο. Ένα παράδειγμα εφαρμογής
      είναι η αυτόματη πλοήγηση ενός οχήματος.
    }
\end{itemize}
Περεταίρω, οι Supervised Learning αλγόριθμοι χωρίζονται σε 2 κατηγορίες, αναλόγα
με την επιθυμητή μορφή της εξόδου του αλγόριθμου ML:
\begin{itemize}
  \item{Ταξινόμησης - Classification: Όταν η έξοδος παίρνει διακριτές τιμές (discrete).}
  \item{Regression: Όταν η έξοδος παίρνει συνεχείς τιμές.}
\end{itemize}

Γενικότερα, οι αλγόριθμοι ML ομαδοποιούνται και ανάλογα με την ομοιότητα τους
σε σχέση με με την λειτουργία που εκτελούνε. Πιο κάτω αναφέρονται οι πιο δημοφιλείς
αλγόριθμοι ML, ομαδοποιημένοι με βάση την λειτουργία τους
\\

\begin{minipage}{0.5\textwidth}

  \textbf{\large Regression}

  Aσχολείται με τη μοντελοποίηση της σχέσης μεταξύ των μεταβλητών που επαναληπτικά ανανεώνονται
  χρησιμοποιώντας ένα μέτρο σφάλματος για τις προβλέψεις που γίνονται από το μοντέλο
  \begin{itemize}
    \setlength\itemsep{0em}
    \item{Ordinary Least Squares Regression (OLSR)}
    \item{Linear Regression}
    \item{Logistic Regression}
    \item{Stepwise Regression}
    \item{Multivariate Adaptive Regression Splines (MARS)}
    \item{Locally Estimated Scatterplot Smoothing (LOESS)}
  \end{itemize}
\end{minipage}
\begin{minipage}{0.5\textwidth}
  \begin{center}
    \includegraphics[width=0.6\textwidth]{./images/chapter3/regression_algorithms.png}
  \end{center}
  %\caption{Αλγόριθμοι Regression}
\end{minipage}

\begin{minipage}{0.5\textwidth}

\textbf{\large Instance-based}

  Αυτές οι μεθόδοι δημιουργούν μία βάση δεδομένων με
  παραδείγματα δεδομένων και συγκρίνουν τα νέα δεδομένα με αυτά που έχουν
  καταχωρηθεί στην βάση δεδομένων χρησιμοποιώντας ένα μέτρο ομοιότητας,
  για την εύρεση της καλύτερης αντιστοιχίας, πιθανοτικά.
  \begin{itemize}
    \setlength\itemsep{0em}
    \item{k-Nearest Neighbour (kNN)}
    \item{Learning Vector Quantization (LVQ)}
    \item{Self-Organizing Map (SOM)}
    \item{Locally Weighted Learning (LWL)}
  \end{itemize}
\end{minipage}
\begin{minipage}{0.5\textwidth}
  \begin{center}
    \includegraphics[width=0.6\textwidth]{./images/chapter3/instance_based_algorithms.png}
  \end{center}
  %\caption{Αλγόριθμοι Regression}
\end{minipage}


\begin{minipage}{0.5\textwidth}  %% Minipage

  \textbf{\large Regularization}

  Χρησιμοποιούνται σαν επεκτάσεις άλλων μεθόδων και
  "τιμωρούν" μοντέλα, βασισμένμενα στην πολυπολοκότητα τους, ευνοώντας έτσι
  απλούστερα μοντέλα τα οποίο είναι παράλληλα καλύτερα στην γενίκευση της
  επίλυσης του εκάστοτε προβλήματος.
  \begin{itemize}
    \setlength\itemsep{0em}
    \item{Ridge Regression}
    \item{Least Absolute Shrinkage and Selection Operator (LASSO)}
    \item{Least-Angle Regression (LARS)}
    \item{Elastic Net}
  \end{itemize}
\end{minipage}
\begin{minipage}{0.5\textwidth}
  \begin{center}
    \includegraphics[width=0.6\textwidth]{./images/chapter3/regularization_algorithms.png}
  \end{center}
  %\caption{Αλγόριθμοι Regression}
\end{minipage}

\begin{minipage}{0.5\textwidth}

  \textbf{\large Dimensionality Reduction}

  Χρησιμοποιούνται για την αφαίρεση σχεδόν ασήμαντης πληροφορίας
  από τα δεδομέναι. Πολλές από τις μεθόδους αυτές χρησιμοποιούνται σαν επεκτάσεις σε μοντέλα επίλυσης
  προβλημάτων regression και classification

  \begin{itemize}
    \setlength\itemsep{0em}
    \item{Principal Component Analysis (PCA)}
    \item{Discriminant Analysis: Linear (LDA), Mixture (MDA), Quadratic (QDA), Flexible (FDA)}
    \item{Principal Component Regression (PCR)}
    \item{Multidimensional Scaling (MDS)}
  \end{itemize}
\end{minipage}
\begin{minipage}{0.5\textwidth}
  \begin{center}
    \includegraphics[width=0.6\textwidth]{./images/chapter3/dimensional_reduction_algorithms.png}
  \end{center}
  %\caption{Αλγόριθμοι Regression}
\end{minipage}

\begin{minipage}{0.5\textwidth}

  \textbf{\large Decision Trees}

  Χρησιμοποιούνται για την κατασκευή μοντέλων
  λήψης αποφάσεων, τα οποία χρησιμοποιούν τις πραγματικές τιμές των
  χαρακτηριστικών των δεδομένων.
  \begin{itemize}
    \setlength\itemsep{0em}
    \item{Classification and Regression Tree (CART)}
    \item{Conditional Decision Trees}
    \item{M5}
  \end{itemize}
\end{minipage}
\begin{minipage}{0.5\textwidth}
  \begin{center}
    \includegraphics[width=0.6\textwidth]{./images/chapter3/decision_tree_algorithms.png}
  \end{center}
  %\caption{Αλγόριθμοι Regression}
\end{minipage}

\begin{minipage}{0.5\textwidth}

  \textbf{\large Bayesian}

  Εφαρμόζουν το θεώρημα του Bayes για την επίλυση τόσο προβλημάτων regression, αλλά και classification
  \begin{itemize}
    \setlength\itemsep{0em}
    \item{Naive Bayes}
    \item{Gaussian Naive Bayes}
    \item{Bayesian Network (BN)}
    \item{Bayesian Belief Network (BBN)}
  \end{itemize}
\end{minipage}
\begin{minipage}{0.5\textwidth}
  \begin{center}
    \includegraphics[width=0.6\textwidth]{./images/chapter3/bayesian_algorithms.png}
  \end{center}
  %\caption{Αλγόριθμοι Regression}
\end{minipage}

\begin{minipage}{0.5\textwidth}

  \textbf{\large Clustering}

  Περιγράφουν τις κλάσεις του προβλήματος
  \begin{itemize}
    \setlength\itemsep{0em}
    \item{k-Means}
    \item{k-Medians}
    \item{Expectation Maximisation (EM)}
    \item{Hierarchical Clustering}
  \end{itemize}
\end{minipage}
\begin{minipage}{0.5\textwidth}
  \begin{center}
    \includegraphics[width=0.6\textwidth]{./images/chapter3/bayesian_algorithms.png}
  \end{center}
  %\caption{Αλγόριθμοι Regression}
\end{minipage}

\begin{minipage}{0.5\textwidth}

  \textbf{\large Artificial Neural Networks (ANN)}

  Μοντέλα εμπνευσμένα από τη δομή ή/και την λειτουργία των βιολογικών νευρωνικών δικτύων.
  Χρησιμοποιούνται στην επίλυση προβλημάτων classification ή/και regression
  \begin{itemize}
    \setlength\itemsep{0em}
    \item{Perceptron}
    \item{Back-Propagation}
    \item{Radial Basis Function Network (RBFN)}
  \end{itemize}
\end{minipage}
\begin{minipage}{0.5\textwidth}
  \begin{center}
    \includegraphics[width=0.6\textwidth]{./images/chapter3/artificial_neural_network_algorithms.png}
  \end{center}
  %\caption{Αλγόριθμοι Regression}
\end{minipage}

\begin{minipage}{0.5\textwidth}

  \textbf{\large Deep Learning (DL)}

  Οι αλγόριθμοι DL είναι η σύγχρονη επέκταση των ANN, τα οποία
  εκμεταλλεύονται την αφθονία υπολογιστικής ισχύς των σύγχρονων υπολογιστικών συστημάτων.
  \begin{itemize}
    \setlength\itemsep{0em}
    \item{Deep Boltzmann Machine (DBM)}
    \item{Deep Belief Networks (DBN)}
    \item{Convolutional Neural Network (CNN)}
    \item{Stacked Auto-Encoders}
    \item{Recurrent Neural Networks (RNN)}
  \end{itemize}%
\end{minipage}
\begin{minipage}{0.5\textwidth}
  \begin{center}
    \includegraphics[width=0.6\textwidth]{./images/chapter3/deep_learning_algorithms.png}
  \end{center}
  %\caption{Αλγόριθμοι Regression}
\end{minipage}
\\\\

Η μορφή της αναπαράστασης των δεδομένων αποτελεί σημαντικό παράγοντα στην
απόδοση των αλγορίθμων ML. Μία αναπαράσταση αποτελείται από χαραχτηριστικά (features).
Για παράδειγμα, ένα χρήσιμο χαρακτηριστικό, στην ταυτοποίηση ομιλιτή, από δεδομένα ήχου,
είναι η εκτίμηση του μεγέθους της φωνητικής έκτασης του ομιλιτή.
Έτσι, πολλά προβλήματα τεχνητής νοημοσύνης, μπορούν να λυθούν
με κατάλληλη σχεδίαση και επιλογή των χαρακτηριστικών, για το συγκεκριμένο
πρόβλημα. Το σύνολο των χαρακτηριστικών αυτών αποτελεί την αναπαράσταση των δεδομένων,
σε ένα πιο υψηλό και αφαιρετικό επίπεδο αντίληψης για τους υπολογιστές, η οποία
στην συνέχεια δίνεται σαν είσοδος σε έναν απλό ML αλγόριθμο, ο οποίος έχει
μάθει να αντιστοιχεί την αναπαράσταση των δεδομένων στην επιθυμητή έξοδο.
    \begin{figure}[!h]
      \centering
      \includegraphics[width=0.7\textwidth]{./images/chapter3/representation_dependency.png}
      \caption[Παράδειγμα διαφορετικών αναπαραστάσεων των δεδομένων]{Παράδειγμα διαφορετικών αναπαραστάσεων των δεδομένων}
      \label{fig:representation_dependency}
    \end{figure}

    Ένα απλό και κατανοητό παράδειγμα, το οποίο δείχνει την εξάρτηση της επίδοσης ενός
    αλγορίθμου ML, από την μορφή της αναπαράστασης που του δίνεται, φαίνεται στο
    \autoref{fig:representation_dependency}. Έστω ότι θέλουμε να
    διαχωρίσουμε τα δεδομένα μας σε δύο κλάσεις, χαράζοντας μία ευθεία
    μεταξύ τους. Αν αναπαραστήσουμε τα δεδομένα στο Καρτεσιανό σύστημα συντεταγμένων (αριστερό διάγραμμα),
    τότε η επίλυση του προβλήματος είναι αδύνατη αφού δεν υπαρχει καμία ευθεία
    που να διαχωρίζει τις δύο κλάσεις. Ωστόσο, αν αναπαραστήσουμε τα δεδομένα
    στο Πολικό σύστημα συντεταγμένων (δεξί διάγραμμα), τότε το πρόβλημα λύνεται
    εύκολα, χαράζοντας μία κάθετη ευθεία, με $r  = a, a \in [r1, r2]$.

    Σε πληθώρα προβλημάτων τεχνητής νοημοσύνης, η επιλογή κατάλληλων χαρακτηριστικών
    είναι δύσκολο και χρονοβόρο έργο. Έστω για παράδειγμα ότι θέλουμε να αναγνωρίσουμε
    πρόσωπα σε εικόνες. Ένα χαρακτηριστικό θα μπορούσε να είναι τα μάτια. Δυστυχώς όμως,
    η αναγνώριση ματιών είναι και αυτή ένα δύσκολο πρόβλημα, αφού δεν μπορεί να
    περιγραφεί πάντα επακριβώς έχοντας σαν δεδομένα τις τιμές των pixel της εικόνας.
    Η γεωμετρική, για παράδειγμα, μορφή των ματιών σε μία εικόνα λήψης εξαρτάται από την
    γωνία λήψης, τον φωτισμό, κτλ.


    %\section{Εισαγωγή στην Επιστήμη της Μηχανικής Μάθησης}
\label{sec:theory_ml}

Τα τελευταία χρόνια, ο κλάδος της Τεχνιτής Νοημοσύνης είναι ένας από τους πιο ραγδαία
αναπτυσσόμενους κλάδους της επιστήμης της πληροφορικής με τεράστιο
ερευνητικό και πρακτικό ενδιαφέρον; διαιρείται σε δύο υπο\_\_\_; την \emph{συμβολική τεχνιτή
νοημοσύνη} και την \emph{υποσυμβολική τεχνητή νοημοσύνη}. Η πρώτη προσπαθεί να
επιλύσει τα προβλήματα χρησιμοποιώντας αλγοριθμικές διαδικασίες, δηλαδή
σύμβολα και λογικούς κανόνες, ενώ η δεύτερη προσπαθεί να αναπαράγει την
ανθρώπινη "ευφυία" μέσα από την χρήση αριθμητικών μοντέλων
που με την σύνθεσή τους προσομοιώνουν την λειτουργία του ανθρώπινου εγκεφάλου
(υπολογιστική νοημοσύνη).
Η ικανότητα ενός νοούμενου (AI) συστήματος, να αποκτά από μόνο του γνώση,
εξάγοντας πρότυπα ή/και χαρακτηριστικά σημεία από τα δεδομένα,
είναι γνωστή ώς \emph{Μηχανική Μάθηση} (ML).
Η εισαγωγή του κλάδου του ML στην επιστήμη των υπολογιστών,
επέτρεψε στους υπολογιστές να μπορούν να αντιμετωπίσουν προβλήματα που εμπλέκουν
την αντίληψη για τον πραγματικό κόσμο και να πέρνουν υποκειμενικές αποφάσεις.

Η χρησιμοποίηση αλγορίθμων ML, επιτρέπουν σε συστήματα AI
να προσαρμόζονται εύκολα σε καινούργια έργα, με ελάχιστη επέμβαση από τον άνθρωπο.
Για παράδειγμα, ένα Νευρωνικό Δίκτυο που έχει εκπεδευτεί να αναγνωρίζει γάτους σε εικόνες,
δεν απαιτεί να σχεδιαστεί και να εκπαιδευτεί από το μηδέν για να έχει την ικανότητα
να αναγνωρίζει και σκύλους.

Πολλά προβλήματα, ενώ μέχρι και πριν μία πεντα-ετία λύνονταν με
"χειρόγραφη", προγραμματισμένη από τον άνθρωπο, γνώση
(hand-written feature extraction), σήμερα χρησιμοποιούνται
αλγόριθμοι ML για την επίλυσή τους. Πιο κάτω παρατίθενται μερικά
παραδείγματα:
\begin{itemize}
  \item{Αναγνώριση ομιλίας - Speech Recognition}
  \item{Μηχανική όραση - Computer Vision}
  \begin{itemize}
    \item{Αναγνώριση αντικειμένων σε εικόνες - Object Recognition}
    \item{Αναγνώριση και εντοπισμός της θέσης αντικειμένων σε εικόνες - Object Detection}
  \end{itemize}
  \item{Αναγνώριση ηλεκτρονικών επιθέσεων στο διαδίκτυο - Cyberattack detection}
  \item{Επεξεργασία φυσικής γλώσσας - Natural Language Processing}
  \begin{itemize}
    \item{Κατανόηση της φυσικής γλώσσας του ανθρώπου - Natural Language Understanding}
    \item{Μοντελοποίηση και χρησιμοποίηση της φυσικής γλώσσας του ανθρώπου από μηχανές - Natural Language Generation}
  \end{itemize}
  \item{Μηχανές αναζήτησης}
\end{itemize}

Η μορφή της αναπαράστασης των δεδομένων αποτελεί σημαντικό παράγοντα στην
απόδοση των αλγορίθμων ML. Μία αναπαράσταση αποτελείτε από χαραχτηριστικά (features).
Για παράδειγμα, ένα χρήσιμο χαρακτηριστικό, στην ταυτοποίηση ομιλιτή, από δεδομένα ήχου,
είναι η εκτίμηση του μεγέθους της φωνητικής έκτασης του ομιλιτή.
Έτσι, πολλά προβλήματα τεχνητής νοημοσύνης, μπορούν να λυθούν
με κατάλληλη σχεδίαση και επιλογή των χαρακτηριστικών, για το συγκεκριμένο
πρόβλημα. Το σύνολο των χαρακτηριστικών αυτών αποτελεί την αναπαράσταση των δεδομένων,
σε ένα πιο υψηλό και αφαιρετικό επίπεδο αντίληψης για τους υπολογιστές, η οποία
στην συνέχεια δίνεται σαν είσοδος σε έναν απλό ML αλγόριθμο, ο οποίος έχει
μάθει να αντιστοιχεί την αναπαράσταση των δεδομένων στην επιθυμητή έξοδο.


    \section{Νευρωνικά Δίκτυα με Βάθος}
\label{sec:theory_dnn}
The area of Neural Networks has originally been primarily inspired by the goal of modeling biological neural systems, but has since diverged and become a matter of engineering and achieving good results in Machine Learning tasks. Nonetheless, we begin our discussion with a very brief and high-level description of the biological system that a large portion of this area has been inspired by.
Resource: http://cs231n.github.io/neural-networks-1/

\subsection{Multilayer Perceptron}

The computations involved in producing an output from an input can be represented by a flow graph: a flow graph is a graph representing a computation, in which each node represents an elementary computation and a value (the result of the computation, applied to the values at the children of that node). Consider the set of computations allowed in each node and possible graph structures and this defines a family of functions. Input nodes have no children. Output nodes have no parents.

    \section{Convolution Neural Networks}
\label{sec:theory_cnn}


\section{Αναγνώριση αντικειμένων με χρήση Νευρωνκών Δικτύων Συνέλιξης}
\label{sec:theory_object_recognition}


