\chapter{Τεχνικές Βαθιάς Μηχανικής Μάθησης και Αναγνώριση Αντικειμένωνν στον χώρο}
\label{chapter:theory}

%Στο κεφάλαιο αυτό, παρουσιάζεται και αναλύεται το θεωρητικό υπόβαθρο,
%στο οποίο στηρίχτηκαν οι υλοποιήσεις, στην παρούσα διπλωματική εργασία.

%Στο Κεφάλαιο \ref{sec:theory_ml}, παρουσιάζονται οι βασικές αρχές
%και τεχνικες \emph{Μηχανικής Μάθησης} (Machine Learning - ML).

Τα τελευταία χρόνια, ο κλάδος της Τεχνιτής Νοημοσύνης είναι ένας από τους πιο ραγδαία
αναπτυσσόμενους κλάδους της επιστήμης της πληροφορικής με τεράστιο
ερευνητικό και πρακτικό ενδιαφέρον; διαιρείται σε δύο υπο\_\_\_; την \emph{συμβολική τεχνιτή
νοημοσύνη} και την \emph{υποσυμβολική τεχνητή νοημοσύνη}. Η πρώτη προσπαθεί να
επιλύσει τα προβλήματα χρησιμοποιώντας αλγοριθμικές διαδικασίες, δηλαδή
σύμβολα και λογικούς κανόνες, ενώ η δεύτερη προσπαθεί να αναπαράγει την
ανθρώπινη "ευφυία" μέσα από την χρήση αριθμητικών μοντέλων
που με την σύνθεσή τους προσομοιώνουν την λειτουργία του ανθρώπινου εγκεφάλου
(υπολογιστική νοημοσύνη).
Η ικανότητα ενός νοούμενου (AI) συστήματος, να αποκτά από μόνο του γνώση,
εξάγοντας πρότυπα ή/και χαρακτηριστικά σημεία από τα δεδομένα,
είναι γνωστή ώς \emph{Μηχανική Μάθηση} (ML).
Η εισαγωγή του κλάδου του ML στην επιστήμη των υπολογιστών,
επέτρεψε στους υπολογιστές να μπορούν να αντιμετωπίσουν προβλήματα που εμπλέκουν
την αντίληψη για τον πραγματικό κόσμο και να πέρνουν υποκειμενικές αποφάσεις.

Η χρησιμοποίηση αλγορίθμων ML, επιτρέπουν σε συστήματα AI
να προσαρμόζονται εύκολα σε καινούργια έργα, με ελάχιστη επέμβαση από τον άνθρωπο.
Για παράδειγμα, ένα Νευρωνικό Δίκτυο που έχει εκπεδευτεί να αναγνωρίζει γάτους σε εικόνες,
δεν απαιτεί να σχεδιαστεί και να εκπαιδευτεί από το μηδέν για να έχει την ικανότητα
να αναγνωρίζει και σκύλους.

Πολλά προβλήματα, ενώ μέχρι και πριν μία μερικά χρόνια λύνονταν με
"χειρόγραφη", προγραμματισμένη από τον άνθρωπο, γνώση
(hand-written feature extraction), σήμερα χρησιμοποιούνται
αλγόριθμοι ML για την επίλυσή τους. Πιο κάτω παρατίθενται μερικά
παραδείγματα:
\begin{itemize}
  \item{Αναγνώριση ομιλίας - Speech Recognition}
  \item{Μηχανική όραση - Computer Vision}
  \begin{itemize}
    \item{Αναγνώριση αντικειμένων σε εικόνες - Object Recognition}
    \item{Αναγνώριση και εντοπισμός της θέσης αντικειμένων σε εικόνες - Object Detection}
  \end{itemize}
  \item{Αναγνώριση ηλεκτρονικών επιθέσεων στο διαδίκτυο - Cyberattack detection}
  \item{Επεξεργασία φυσικής γλώσσας - Natural Language Processing}
  \begin{itemize}
    \item{Κατανόηση της φυσικής γλώσσας του ανθρώπου - Natural Language Understanding}
    \item{Μοντελοποίηση και χρησιμοποίηση της φυσικής γλώσσας του ανθρώπου από μηχανές - Natural Language Generation}
  \end{itemize}
  \item{Μηχανές αναζήτησης}
\end{itemize}

Η μορφή της αναπαράστασης των δεδομένων αποτελεί σημαντικό παράγοντα στην
απόδοση των αλγορίθμων ML. Μία αναπαράσταση αποτελείτε από χαραχτηριστικά (features).
Για παράδειγμα, ένα χρήσιμο χαρακτηριστικό, στην ταυτοποίηση ομιλιτή, από δεδομένα ήχου,
είναι η εκτίμηση του μεγέθους της φωνητικής έκτασης του ομιλιτή.
Έτσι, πολλά προβλήματα τεχνητής νοημοσύνης, μπορούν να λυθούν
με κατάλληλη σχεδίαση και επιλογή των χαρακτηριστικών, για το συγκεκριμένο
πρόβλημα. Το σύνολο των χαρακτηριστικών αυτών αποτελεί την αναπαράσταση των δεδομένων,
σε ένα πιο υψηλό και αφαιρετικό επίπεδο αντίληψης για τους υπολογιστές, η οποία
στην συνέχεια δίνεται σαν είσοδος σε έναν απλό ML αλγόριθμο, ο οποίος έχει
μάθει να αντιστοιχεί την αναπαράσταση των δεδομένων στην επιθυμητή έξοδο.
\begin{figure}[!h]
  \centering
  \includegraphics[width=0.7\textwidth]{./images/chapter3/representation_dependency.png}
  \caption[Παράδειγμα διαφορετικών αναπαραστάσεων των δεδομένων]{Παράδειγμα διαφορετικών αναπαραστάσεων των δεδομένων}
  \label{fig:representation_dependency}
\end{figure}

Ένα απλό και κατανοητό παράδειγμα που δείχνει την εξάρτηση της επίδοσης ενός
αλγορίθμου ML, από την μορφή της αναπαράστασης που του δίνεται, φαίνεται στο
Σχήμα \ref{fig:representation_dependency}. Έστω ότι θέλουμε να
κατηγοριοποιήσουμε τα δεδομένα μας σε δύο κλάσεις, χαράζοντας μία ευθεία
μεταξύ τους. Αν αναπαραστήσουμε τα δεδομένα στο Καρτεσιανό σύστημα συντεταγμένων (αριστερό διάγραμμα),
τότε η επίλυση του προβλήματος είναι αδύνατη αφού δεν υπαρχει καμία ευθεία
που να διαχωρίζει τις δύο κλάσεις. Ωστόσο, αν αναπαραστήσουμε τα δεδομένα
στο Πολικό σύστημα συντεταγμένων (δεξί διάγραμμα), τότε το πρόβλημα λύνεται
εύκολα, χαράζοντας μία κάθετη ευθεία, με $r \in [r1, r2], \theta \in [0, 2\pi]$.


%\section{Εισαγωγή στην Επιστήμη της Μηχανικής Μάθησης}
\label{sec:theory_ml}

Τα τελευταία χρόνια, ο κλάδος της Τεχνιτής Νοημοσύνης είναι ένας από τους πιο ραγδαία
αναπτυσσόμενους κλάδους της επιστήμης της πληροφορικής με τεράστιο
ερευνητικό και πρακτικό ενδιαφέρον; διαιρείται σε δύο υπο\_\_\_; την \emph{συμβολική τεχνιτή
νοημοσύνη} και την \emph{υποσυμβολική τεχνητή νοημοσύνη}. Η πρώτη προσπαθεί να
επιλύσει τα προβλήματα χρησιμοποιώντας αλγοριθμικές διαδικασίες, δηλαδή
σύμβολα και λογικούς κανόνες, ενώ η δεύτερη προσπαθεί να αναπαράγει την
ανθρώπινη "ευφυία" μέσα από την χρήση αριθμητικών μοντέλων
που με την σύνθεσή τους προσομοιώνουν την λειτουργία του ανθρώπινου εγκεφάλου
(υπολογιστική νοημοσύνη).
Η ικανότητα ενός νοούμενου (AI) συστήματος, να αποκτά από μόνο του γνώση,
εξάγοντας πρότυπα ή/και χαρακτηριστικά σημεία από τα δεδομένα,
είναι γνωστή ώς \emph{Μηχανική Μάθηση} (ML).
Η εισαγωγή του κλάδου του ML στην επιστήμη των υπολογιστών,
επέτρεψε στους υπολογιστές να μπορούν να αντιμετωπίσουν προβλήματα που εμπλέκουν
την αντίληψη για τον πραγματικό κόσμο και να πέρνουν υποκειμενικές αποφάσεις.

Η χρησιμοποίηση αλγορίθμων ML, επιτρέπουν σε συστήματα AI
να προσαρμόζονται εύκολα σε καινούργια έργα, με ελάχιστη επέμβαση από τον άνθρωπο.
Για παράδειγμα, ένα Νευρωνικό Δίκτυο που έχει εκπεδευτεί να αναγνωρίζει γάτους σε εικόνες,
δεν απαιτεί να σχεδιαστεί και να εκπαιδευτεί από το μηδέν για να έχει την ικανότητα
να αναγνωρίζει και σκύλους.

Πολλά προβλήματα, ενώ μέχρι και πριν μία πεντα-ετία λύνονταν με
"χειρόγραφη", προγραμματισμένη από τον άνθρωπο, γνώση
(hand-written feature extraction), σήμερα χρησιμοποιούνται
αλγόριθμοι ML για την επίλυσή τους. Πιο κάτω παρατίθενται μερικά
παραδείγματα:
\begin{itemize}
  \item{Αναγνώριση ομιλίας - Speech Recognition}
  \item{Μηχανική όραση - Computer Vision}
  \begin{itemize}
    \item{Αναγνώριση αντικειμένων σε εικόνες - Object Recognition}
    \item{Αναγνώριση και εντοπισμός της θέσης αντικειμένων σε εικόνες - Object Detection}
  \end{itemize}
  \item{Αναγνώριση ηλεκτρονικών επιθέσεων στο διαδίκτυο - Cyberattack detection}
  \item{Επεξεργασία φυσικής γλώσσας - Natural Language Processing}
  \begin{itemize}
    \item{Κατανόηση της φυσικής γλώσσας του ανθρώπου - Natural Language Understanding}
    \item{Μοντελοποίηση και χρησιμοποίηση της φυσικής γλώσσας του ανθρώπου από μηχανές - Natural Language Generation}
  \end{itemize}
  \item{Μηχανές αναζήτησης}
\end{itemize}

Η μορφή της αναπαράστασης των δεδομένων αποτελεί σημαντικό παράγοντα στην
απόδοση των αλγορίθμων ML. Μία αναπαράσταση αποτελείτε από χαραχτηριστικά (features).
Για παράδειγμα, ένα χρήσιμο χαρακτηριστικό, στην ταυτοποίηση ομιλιτή, από δεδομένα ήχου,
είναι η εκτίμηση του μεγέθους της φωνητικής έκτασης του ομιλιτή.
Έτσι, πολλά προβλήματα τεχνητής νοημοσύνης, μπορούν να λυθούν
με κατάλληλη σχεδίαση και επιλογή των χαρακτηριστικών, για το συγκεκριμένο
πρόβλημα. Το σύνολο των χαρακτηριστικών αυτών αποτελεί την αναπαράσταση των δεδομένων,
σε ένα πιο υψηλό και αφαιρετικό επίπεδο αντίληψης για τους υπολογιστές, η οποία
στην συνέχεια δίνεται σαν είσοδος σε έναν απλό ML αλγόριθμο, ο οποίος έχει
μάθει να αντιστοιχεί την αναπαράσταση των δεδομένων στην επιθυμητή έξοδο.


\section{Νευρωνικά Δίκτυα με Βάθος}
\label{sec:theory_dnn}
The area of Neural Networks has originally been primarily inspired by the goal of modeling biological neural systems, but has since diverged and become a matter of engineering and achieving good results in Machine Learning tasks. Nonetheless, we begin our discussion with a very brief and high-level description of the biological system that a large portion of this area has been inspired by.
Resource: http://cs231n.github.io/neural-networks-1/

\subsection{Multilayer Perceptron}

The computations involved in producing an output from an input can be represented by a flow graph: a flow graph is a graph representing a computation, in which each node represents an elementary computation and a value (the result of the computation, applied to the values at the children of that node). Consider the set of computations allowed in each node and possible graph structures and this defines a family of functions. Input nodes have no children. Output nodes have no parents.

\section{Convolution Neural Networks}
\label{sec:theory_cnn}


\section{Αναγνώριση αντικειμένων με χρήση Νευρωνκών Δικτύων Συνέλιξης}
\label{sec:theory_object_recognition}


n-tk1-labelled
