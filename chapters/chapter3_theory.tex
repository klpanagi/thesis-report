\chapter{Τεχνικές Βαθιάς Μηχανικής Μάθησης και Αναγνώριση Αντικειμένων στον χώρο}
\label{chapter:theory}

Τα τελευταία χρόνια, ο κλάδος της Τεχνητής Νοημοσύνης είναι ένας από τους πιο ραγδαία
αναπτυσσόμενους κλάδους της επιστήμης της πληροφορικής με τεράστιο
ερευνητικό και πρακτικό ενδιαφέρον; διαιρείται σε δύο υπο\_\_\_; την \emph{συμβολική τεχνητή
νοημοσύνη} και την \emph{υποσυμβολική τεχνητή νοημοσύνη}. Η πρώτη προσπαθεί να
επιλύσει τα προβλήματα χρησιμοποιώντας αλγοριθμικές διαδικασίες, δηλαδή
σύμβολα και λογικούς κανόνες, ενώ η δεύτερη προσπαθεί να αναπαράγει την
ανθρώπινη "ευφυΐα" μέσα από την χρήση αριθμητικών μοντέλων
που με την σύνθεσή τους προσομοιώνουν την λειτουργία του ανθρώπινου εγκεφάλου
(υπολογιστική νοημοσύνη).

Η ικανότητα ενός νοούμενου (AI) συστήματος, να αποκτά από μόνο του γνώση,
εξάγοντας πρότυπα ή/και χαρακτηριστικά σημεία από τα δεδομένα,
είναι γνωστή ως \emph{Μηχανική Μάθηση} (ML).

Στο κεφάλαιο αυτό, παρουσιάζονται και αναλύονται τεχνικές και αλγόριθμοι
\emph{Μηχανικής Μάθησης}, με επίκεντρο τα νευρωνικά δίκτυα με βάθος (Deep Neural Networks - DNN).
Στόχος είναι ο αναγνώστης να αντιληφθεί και να κατανοήσει τις βασικές αρχές και
λειτουργίες των νευρωνικών αυτών δικτύων, αφού είναι οι βάσεις για την περαιτέρω
μελέτη των νευρωνικών δικτύων συνέλιξης (CNN) και των εφαρμογών αυτών στο
πρόβλημα της αναγνώρισης και εντοπισμού αντικειμένων σε εικόνες.

\input{./chapters/chapter3_theory/section0_dl.tex}
\section{Νευρωνικά Δίκτυα με Βάθος}
\label{sec:theory_dnn}
The area of Neural Networks has originally been primarily inspired by the goal of modeling biological neural systems, but has since diverged and become a matter of engineering and achieving good results in Machine Learning tasks. Nonetheless, we begin our discussion with a very brief and high-level description of the biological system that a large portion of this area has been inspired by.
Resource: http://cs231n.github.io/neural-networks-1/

\subsection{Multilayer Perceptron}

The computations involved in producing an output from an input can be represented by a flow graph: a flow graph is a graph representing a computation, in which each node represents an elementary computation and a value (the result of the computation, applied to the values at the children of that node). Consider the set of computations allowed in each node and possible graph structures and this defines a family of functions. Input nodes have no children. Output nodes have no parents.

\section{Convolution Neural Networks}
\label{sec:theory_cnn}


