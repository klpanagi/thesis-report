\chapter{Υλοποιήσεις}
\label{chapter:implementations}

Η υλοποίηση στην οποία στηρίζεται η παρούσα διπλωματική εργασία είναι το
YOLO Net \cite{DBLP:journals/corr/RedmonDGF15}, το οποίο είναι ένα μοντέλο CNN για ταυτόχρονη αναγνώριση και εντοπισμό
αντικειμένων σε εικόνες.

Όπως αναφέρθηκε στο \autoref{chapter:tools} στόχος είναι η υλοποίηση και ανάπτυξη
του ΥΟLO Net πάνω στο ενσωματωμένο σύστημα Jetson TK1 της NVIDIA.

Θεωρούμε σημαντικό να αναφέρουμε σε αυτό το σημείο ότι προτού γίνει υλοποίηση
του συγκεκριμένου μοντέλου CNN πάνω στο Jetson TK1, χρειάστηκε να εγκαταστήσουμε
τα περισσότερα εργαλεία κατευθείαν πάνω στο μηχάνημα (build from source), για να αποφύγουμε
την εγκατάστασή pre-compiled λογισμικού. Ο λόγος, όπως και θα φανεί στο \autoref{chapter:experiments},
είναι ότι η απόδοση των συγκεκριμένων εργαλείων εξαρτάται τόσο από τις
βιβλιοθήκες που θα χρησιμοποιηθούν για πράξεις γραμμικής άλγεβρας,
όσο και από τις βελτιστοποιήσεις που γίνονται από τον μεταγλωττιστή κατά την
διάρκεια της μεταγλώττισης του εκάστοτε πηγαίου κώδικά.

\section{YOLO implementation with keras DNN framework}
\label{sec:implementations_yolo}


TODO!!

\section{Βελτιστοποίηση των εργαλείων λογισμικού στο Jetson TK1}

Ένα μεγάλο μέρος της συνεισφοράς της παρούσας διπλωματική εργασίας
ήταν να γίνουν βελτιστοποιήσεις σε επίπεδο λογισμικού των εργαλείων που χρησιμοποιήθηκαν
για το συγκεκριμένο μηχάνημα.

\label{sec:implementations_jetson}


