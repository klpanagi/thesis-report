\section{Περιγραφή του Προβλήματος}
\label{section:problem_description}

Παρόλο που τα CΝΝ είναι ικανά να δώσουν λύσεις με μεγάλη ακρίβεια, απαιτούν
μεγάλο όγκο επεξεργαστικής ισχύος, τόσο για την εκπαίδευσή τους, όσο και για την
εκτέλεση ενός πειράματος, ιδιαίτερα όταν το πρόβλημα για το οποίο έχουν
σχεδιαστεί να δώσουν λύση είναι περίπλοκο. Η απαίτηση αυτή προέρχεται από το βάθος %δεν έχεις αναφέρει πιο πανω για βάθος οπότε ο άλλος δε θα καταλαβει τι λές. πες το λίγο αλλιώς, πιο σύνθετο πχ
των μοντέλων CNN για αναγνώριση και εντοπισμό αντικειμένων σε εικόνες.
Για παράδειγμα, ένα από τα πρώτα μοντέλα CNN το οποίο σχεδιάστηκε για την
αναγνώριση αντικειμένων σε εικόνες, αποτελείται από 16 επίπεδα (AlexNet)
και έχει εξήντα εκατομμύρια (60,000,000) παραμέτρους και
εξακόσιες πενήντα χιλιάδες (650,000) νευρώνες από τους οποίους οι περισσότεροι
εκτελούν πράξεις συνέλιξης. Ο Alex Krizhevsky απέδειξε το 2012 ότι η χρήση
σύγχρονων GPU για την εκτέλεση πράξεων συνέλιξης, φέρει σαν αποτέλεσμα την
εκπαίδευση μοντέλων CNN σε χρόνους εώς και δύο τάξεις μεγέθους πιο κάτω σε σχέση
με έναν ισχυρό επεξεργαστή \cite{NIPS2012_4824}.
Ο χρόνος εκτέλεσης ενός πειράματος του Δικτύου AlexNet
έχει μετρηθεί στα 7.39 δευτερόλεπτα σε οκταπύρηνο επεξεργαστή Haskwell @2.9Ghz
και στα 0.71 δευτερόλεπτα σε μονάδα GPU, Nvidia K520 \cite{abuzaidoptimizing}.

Είναι ιδιαίτερα σημαντικό, ένα ρομπότ να μπορεί να αντιλαμβάνεται το περιβάλλον
του.
Αυτό περιλαμβάνει την αναγνώριση ανθρώπων, ζώων, και γενικότερα αντικειμένων. Ωστόσο,
θέλουμε τα ρομπότ να είναι και όσο πιο "ελκυστικά" γίνεται στον άνθρωπο ή/και μικρότερα,
ανάλογα με το task που επιθυμούμε να εκτελέσουν.
Αυτό, φέρει σαν αποτέλεσμα να μην μπορούμε να τοποθετήσουμε ογκώδη, άρα με μεγάλη
υπολογιστική ισχύ, υπολογιστικά συστήματα στο σώμα των ρομποτικών συστημάτων σε όλες
τις περιπτώσεις.

Παρόλο που σήμερα έχουν σχεδιαστεί μοντέλα CNN, τα οποία έχουν την δυνατότητα να
αναγνωρίσουν και να εντοπίσουν αντικείμενα από χιλιάδες, αν όχι και περισσότερες,
κλάσεις, ο χρόνος που απαιτείται για να κατηγοριοποιήσει αντικείμενα σε μία εικόνα
είναι αρκετά μεγάλος (της τάξης των μερικών δευτερολέπτων σε σύγχρονους υπολογιστές).
Αυτό κάνει την χρήση των CNN σε εφαρμογές πραγματικού χρόνου, όπως για παράδειγμα
στην ρομποτική, ακατάλληλη.
Ωστόσο, η επιστημονική κοινότητα σήμερα προσπαθεί να δώσει λύσεις στο συγκεκριμένο
πρόβλημα εστιάζοντας το ενδιαφέρον στην εξέλιξη των ενσωματωμένων συστημάτων
και την σχεδίαση γρήγορου, σε χρόνο εκτέλεσης, λογισμικού για υλοποιήσεις μοντέλων CNN τα οποία
εκμεταλλεύονται κυρίως την υπολογιστική ισχύ των μονάδων GPU, αλλά και άλλων
πολυπύρηνων επεξεργαστικών μονάδων.
