\section{Περιγραφή του Προβλήματος}
\label{section:problem_description}

Παρόλο που τα CΝΝ είναι ικανά να δώσουν λύσεις με μεγάλη ακρίβεια, απαιτούν
μεγάλο όγκο επεξεργαστικής ισχύς, τόσο για την εκπαίδευσή τους, όσο και για την
εκτέλεση ενός πειράματος, όταν το πρόβλημα για το οποίο έχουν
σχεδιαστεί να δώσουν λύση είναι περίπλοκο. Η απαίτηση αυτή προέρχεται από το βάθος
των μοντέλων CNN για αναγνώριση και εντοπισμό αντικειμένων σε εικόνες.
Για παράδειγμα, ένα από τα πρώτα μοντέλα CNN το οποίο σχεδιάστηκε για την
αναγνώριση αντικειμένων σε εικόνες, αποτελείτε από 16 επίπεδα (AlexNet)
και έχει εξήντα εκατομμύρια (60,000,000) παραμέτρους και
εξακόσες-πενήντα χιλιάδες (650,000) νευρώνες από τους οποίους οι περισσότεροι
εκτελούν πράξεις συνέλιξης. Ο Alex Krizhevsky απέδειξε το 2012 ότι η χρησιμοποίηση
των σύγρονων GPU για την εκτέλεση πράξεων συνέλιξης, φέρει σαν αποτέλεσμα την
εκαπάίδευση μοντέλων CNN σε χρόνους εώς και δύο τάξεις μεγέθους πιο κάτω σε σχέση
με έναν ισχυρό επεξεργαστή \cite{NIPS2012_4824}. Ο χρόνος εκτέλεσης ενός πειράματος του Δικτύου AlexNet
έχει μετρηθεί στα 7.39 δευτερόλεπτα σε οκτα-πύρηνο επεξεργαστή Haskwell @2.9Ghz
και στα 0.71 δευτερόλεπτα σε μονάδα GPU, Nvidia K520 \cite{abuzaidoptimizing}.

Είναι ιδιαίτερα σημαντικό, ένα ρομπότ να μπορεί να αντιλαμβάνεται το περιβάλλον
του; να μπορεί να αναγνωρίζει ανθρώπους, ζώα, αντικείμενα γενικότερα. Ωστόσο,
θέλουμε τα ρομπότ να είναι και όσο πιο "ελκιστικά" γίνεται στον άνθρωπο ή/και μικρότερα.
Αυτό, φέρει σαν αποτέλεσμα να μην μπορούμε να τοποθετήσουμε ογκώδη, άρα με μεγάλη
υπολογιστική ισχύ, υπολογιστικά συστήματα κατευθείαν πάνω στα ρομπότ.

Παρόλο που σήμερα έχουν σχεδιαστεί μοντέλα CNN, τα οποία έχουν την δυνατότητα να
αναγνωρίσουν και να κατηγοριοποιήσουν αντικείμενα από χιλιάδες, αν όχι και περισσότερες,
κλάσεις, ο χρόνος που απαιτείται για να κατηγοριοποιήσει αντικείμενα σε μία εικόνα
είναι αρκετά μεγάλος, της τάξης των μερικών δευτερολέπτων σε σύνχρονους υπολογιστές.
Αυτό κάνει την χρήση CNN σε εφαρμογές πραγματικού χρόνου, όπως για παράδειγμα
στην ρομποτική, ακατάλληλη.
Ωστόσο, η επιστημονική κοινότητα σήμερα προσπαθεί να δώσει λύσεις στο συγκεκριμένο
πρόβλημα εστιάζοντας το ενδιαφέρον στην εξέλιξη των ενσωματωμένων συστημάτων
και την σχεδίαση γρήγορων λογισμικών για υλοποιήσεις μοντέλων DNN τα οποία
εκμεταλλεύονται κυρίως την υπολογιστική ισχύ των μονάδων GPU, αλλά και άλλων
πολυπύρηνων επεξεργαστικών μονάδων.
