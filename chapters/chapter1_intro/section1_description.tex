\section{Περιγραφή του Προβλήματος}
\label{section:problem_description}

Ο μεγάλος χρόνος που χρειάζεται ένα CNN, τόσο για να κατηγοριοποιήσει, όσο και
για να εντοπίσει, ένα αντικείμενο το κάνει ακατάλληλο για χρήση σε εφαρμογές πραγματικού
χρόνου. Αυτό συμβαίνει λόγω του "βάθους" που έχουνε τα νευρωνικά αυτά δίκτυα.
Είναι ιδιαίτερα σηματνικό, ένα ρομπότ να μπορεί να αντιλαμβάνεται το περιβάλλον
του; να μπορεί να αναγνωρίζει ανθρώπους, ζώα, αντικείμενα γενικότερα. Ωστόσο,
θέλουμε τα ρομπότ να είναι και όσο πιο "ελκιστικά" γίνεται στον άνθρωπο ή/και μικρότερα.
Αυτό, φέρει σαν αποτέλεσμα να μην μπορούμε να φορέσουμε μεγάλα, με μεγάλη,
υπολογιστική ισχύ, υπολογιστικά συστήματα, κατευθείαν πάνω στα ρομπότ.

