\chapter{Επισκόπηση της Ερευνητικής Περιοχής}
\label{chapter:sota}
Τόσο η αναγνώριση αντικειμένων (object recognition) όσο και ο εντοπισμός
της θέσης των αντικειμένων αυτών (detection/localization) σε εικόνες,
είναι μία ερευνητική περιοχή με τεράστιο ενδιαφέρον,
η οποία απασχολεί πληθώρα ερευνητών. Η επιστήμη της Μηχανικής Όρασης (Computer Vision - ML), %ML?? Machine learning?
στοχεύει στο να δώσει λύσεις στα συγκεκριμένα προβλήματα, εισάγοντας αναλυτικά
ή και πιθανοτικά μαθηματικά μοντέλα.

Ο κλάδος της Βαθιάς Μηχανικής Μάθησης (Deep Learning - DL) \cite{Goodfellow-et-al-2016-Book},
ανάγει το πρόβλημα της εύρεσης χαρακτηριστικών σημείων για την αναγνώριση αντικειμένων,
στην εκμάθηση αναπαραστάσεων \cite{bengio2013representation},
με την χρήση Νευρωνικών Δικτύων Συνέλιξης (Convolutional Neural Networks - CNN). %έχεις δώσει το ακρωνύμιο και πιο πάνω
Οι πρώτες εφαρμογές Νευρωνικών Δικτύων Συνέλιξης, για την αναγνώριση αντικειμένων
σε εικόνες, αναπτύχθηκαν το 1990 από τον Yann LeCun.
Η πιο γνωστή και επιτυχής είναι το δίκτυο LeNet \cite{lecun1998gradient}, το οποίο
χρησιμοποιήθηκε για την αναγνώριση ψηφίων σε εικόνες.
Ωστόσο, η εισαγωγή των CNN στον κλάδο της Μηχανικής Όρασης έγινε το 2012 με
την ανάπτυξη του δικτύου AlexNet \cite{NIPS2012_4824}, από τους Alex Krizhevsky,
Ilya Sutskever και Geoffrey E. Hinton. To δίκτυο AlexNet χρησιμοποιήθηκε
στον διαγωνισμό ImageNet ILSVRC challenge, το 2012, κερδίζοντας με διαφορά
10,9\%, στο σφάλμα αναγνώρισης αντικειμένων σε σύνολο 1000 κλάσεων.
Με την εμφάνιση του δικτύου AlexNet, η ερευνητική κοινότητα ξεκίνησε να
πιστεύει στην αποτελεσματικότητα των Νευρωνικών Δικτύων Συνέλιξης σε εφαρμογές αναγνώρισης
αντικειμένων σε εικόνες. Συνέχεια στο έργο του Alex Krizhevsky δόθηκε το 2013, 
αναπτύσσοντας το ZF-Net \cite{DBLP:journals/corr/ZeilerF13}, το οποίο είναι
βασισμένο στην αρχιτεκτονική του δικτύου AlexNet. Μέχρι σήμερα, έχουν σχεδιαστεί
και αναπτυχθεί διάφορα μοντέλα Νευρωνικών Δικτύων Συνέλιξης για
αναγνώριση αντικειμένων, με πιο πρόσφατο το ResNet ,
το οποίο αναπτύχθηκε από τον Kaiming He \cite{DBLP:journals/corr/HeZRS15}. 
Το ResNet (Residual Network) έχει την
ιδιαιτερότητα απουσίας πλήρως συνδεδεμένων επιπέδων και είναι από τα πιο δημοφιλή
μοντέλα που εφαρμόζονται σε πρακτικά προβλήματα αναγνώρισης αντικειμένων σε
εικόνες \cite{DBLP:journals/corr/HeZR016}.

Τα προαναφερθέντα μοντέλα Νευρωνικών Δικτύων Συνέλιξης δίνουν
λύσεις μόνο στο πρόβλημα της αναγνώρισης και όχι
του εντοπισμού της θέσης των αντικειμένων αυτών.
Το 2013, ερευνητές εργαζόμενοι στην Google Inc., σχεδίασαν και υλοποίησαν ένα
μοντέλο Νευρωνικού Δικτύου το οποίο δίνει λύση στο πρόβλημα της ταυτόχρονης
αναγνώρισης και εντοπισμού αντικειμένων πάνω σε πλαίσια εικόνων \cite{szegedy2013deep}.
Το μοντέλο αυτό, που φέρει το όνομα DetectorNet, είναι ομαδική εργασία των
Christian Szegedy, Alexander Toshev και Dumitru Erhan. Το μοντέλο αυτό είναι
πιθανοτικό αφού "προβλέπει" τις οριοθετημένες θέσεις για διάφορες κλάσεις
αντικειμένων στον πλαίσιο μίας εικόνας. Ωστόσο, ένα βασικό μειονέκτημα του DetectorNet
που το κάνει ακατάλληλο για εφαρμογή σε προβλήματα σχεδόν πραγματικού χρόνου (όπως
για παράδειγμα σε ένα ρομποτικό σύστημα), είναι οι τεράστιες απαιτήσεις του σε πόρους
και χρόνο.
\\

\textbf{TODO: A few words about the applications of DNN models in robotics!!!!}

%\section{Νευρωνικά Δίκτυα με Βάθος}
\label{sec:theory_dnn}
The area of Neural Networks has originally been primarily inspired by the goal of modeling biological neural systems, but has since diverged and become a matter of engineering and achieving good results in Machine Learning tasks. Nonetheless, we begin our discussion with a very brief and high-level description of the biological system that a large portion of this area has been inspired by.
Resource: http://cs231n.github.io/neural-networks-1/

\subsection{Multilayer Perceptron}

The computations involved in producing an output from an input can be represented by a flow graph: a flow graph is a graph representing a computation, in which each node represents an elementary computation and a value (the result of the computation, applied to the values at the children of that node). Consider the set of computations allowed in each node and possible graph structures and this defines a family of functions. Input nodes have no children. Output nodes have no parents.

%\section{Convolution Neural Networks}
\label{sec:theory_cnn}


