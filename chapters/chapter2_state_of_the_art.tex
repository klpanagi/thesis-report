\chapter{Επισκόπηση της Ερευνητικής Περιοχής - State of the Art}
\label{chapter:sota}
Τόσο η αναγνώριση αντικειμένων (object recognition) όσο και η χοροτοποθέτηση (detection/localozation) τους,
στο αντίστοιχο σύστημα συντεταγμένων μίας εικόνας, είναι μία ερευνητική περιοχή με τεράστιο ενδιαφέρον και η οποία
απασχολεί πληθώρα ερευνητών. Η επιστήμη της Μηχανικής Όρασης (Machine Learning), στοχεύει στο να δώσει λύσεις
στα συγκεκριμένα προβλήματα, εισάγοντας μαθηματικά μοντέλα, τόσο αναλυτικά, όσο και πιθανοτικά.
Η δυνατότητα ενός "ευφυούς" ρομποτικού συστήματος, να αντιλαμβάνεται
το περιβάλλον του, είναι απαραίτητη ικανότητα που πρέπει να διαθέτει.

O κλάδος της Βαθιάς Μηχανικής Εκμάθησης \cite{Goodfellow-et-al-2016-Book},
ανάγει το πρόβλημα της έυρεσης χαρακτηριστικών σημείων, για την αναγνριση αντικειμένων,
στην εκμάθηση αναπαραστάσεων \cite{bengio2013representation}, με την χρήση Νευρωνικών Δικτύων Συνέλιξης (CNN) \cite{}.
Όπως έγινε γνωστό στο \autoref{chapter:intro} η συγκεκριμένη διπλωματική εργασία
καταπιάνεται με τα πρόβλημα της αναγνώρισης και χοροτοποθέτησης αντικειμένων σε πραγματικό χρόνο (Real-Time Object Detection), με χρήση Νευρωνικών Δικτύων Συνέλιξης (CNN).
Οι ερευνητικές περιοχές στις οποίες εστιάζει είναι:

\begin{itemize}
  \item{Deep Learning \& Deep Neural Networks}
  \item{Convolutional Neural Networks}
  \item{CNN architectures for Real-Time object detection}
\end{itemize}
...

%\section{Νευρωνικά Δίκτυα με Βάθος}
\label{sec:theory_dnn}
The area of Neural Networks has originally been primarily inspired by the goal of modeling biological neural systems, but has since diverged and become a matter of engineering and achieving good results in Machine Learning tasks. Nonetheless, we begin our discussion with a very brief and high-level description of the biological system that a large portion of this area has been inspired by.
Resource: http://cs231n.github.io/neural-networks-1/

\subsection{Multilayer Perceptron}

The computations involved in producing an output from an input can be represented by a flow graph: a flow graph is a graph representing a computation, in which each node represents an elementary computation and a value (the result of the computation, applied to the values at the children of that node). Consider the set of computations allowed in each node and possible graph structures and this defines a family of functions. Input nodes have no children. Output nodes have no parents.

%\section{Convolution Neural Networks}
\label{sec:theory_cnn}


