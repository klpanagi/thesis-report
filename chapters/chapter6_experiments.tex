\chapter{Πειράματα - Αποτελέσματα}
\label{chapter:experiments}

Τα πειράματα χωρίζονται σε δύο φάσεις. Στη πρώτη φάση τα CNN εκτελέστηκαν
σε έναν υπολογιστή με επεξεργαστή \emph{Intel i7 6700} και χωρητικότητα μνήμης
16GB. Στην συνέχεια, με βάση τα αποτελέσματα των πειραμάτων που λάβαμε στην πρώτη φάση επιλέχθηκαν
τα αντίστοιχα μοντέλα CNN για εκτέλεση πάνω στο ενσωματωμένο σύστημα
Jetson TK1. Οι παράμετροι αξιολόγησης των CNN ήταν οι απαιτήσεις σε μνήμη και ο χρόνος εκτέλεσης
της διαδικασίας προς-τα-εμπρός εκτέλεσης (forward propagation).

Για λόγους πληρότητας, πιο κάτω παρουσιάζονται τα βασικά χαρακτηριστικά
των δύο συστημάτων:
\\

\begin{tabular}{ | l | l | l | l | l | l | }
  \hline
  \rowcolor{Gray}
  & \# πυρήνων & \# νημάτων & Συχν. ρολογιού & Κατ. Ισχύος & Μνήμη Cache \\
  \hline
  i7 & 4 & 8 & 3.4 Ghz & ~65Watts & 8MΒ L3\\
  \hline
  TK1 & 4 & 4 & 2.32 & ~11Watts & 2MB L2 \\
  \hline
\end{tabular}
\\\\
Ο αριθμός νημάτων (8) που υποστηρίζει ο επεξεργαστής i7-6500 αναφέρεται στην
υποστήριξη της τεχνολογίας \emph{Hyper-Threading}\footnote{\url{http://www.intel.com/content/www/us/en/architecture-and-technology/hyper-threading/hyper-threading-technology.html}}.
Σημαντικό επίσης να αναφέρουμε ότι η τιμή της κατανάλωσης ισχύος για την
πλατφόρμα Jetson TK1, αντιστοιχεί στην μέγιστη λειτουργία των μονάδων
CPU και GPU.

\section{Πρώτη Φάση Πειραμάτων}
\label{sec:experiments_phase1}

TODO!!

\section{Δεύτερη Φάση Πειραμάτων}
\label{sec:experiments_phase2}

TODO!!

