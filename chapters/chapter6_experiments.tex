\chapter{Πειράματα - Αποτελέσματα}
\label{chapter:experiments}

Τα πειράματα χωρίζονται σε δύο φάσεις. Στη πρώτη φάση τα CNN εκτελέστηκαν
σε έναν υπολογιστή με επεξεργαστή \emph{Intel i7 6700} και χωρητικότητα μνήμης
16GB. Στην συνέχεια, με βάση τα αποτελέσματα που λήφθηκαν στην πρώτη φάση των πειραμάτων, επιλέχθηκαν
τα αντίστοιχα μοντέλα CNN για να εκτελεστούν πάνω στην πλατφόρμα
Jetson TK1. Οι παράμετροι αξιολόγησης ήταν οι απαιτήσεις σε μνήμη και ο χρόνος εκτέλεσης
της διαδικασίας προς-τα-εμπρός περάσματος (forward propagation).

Για λόγους πληρότητας, πιο κάτω παρουσιάζονται τα βασικά χαρακτηριστικά
των δύο συστημάτων:
\\

\begin{table}
\small
\begin{tabular}{ | l | l | l | l | l | l | l | }
  \hline
  \rowcolor{Gray}
  Μηχάνημα & \# πυρήνων & \# νημάτων & clock & Κατ. Ισχύος & RAM & Cache \\
  \hline
  Host PC & 4 & 8 & 3.4 Ghz & ~65Watts & 16GB & 8MΒ L3 \\
  \hline
  TK1 & 4 & 4 & 2.32 & ~11Watts & 2GB (shared) & 2MB L2 \\
  \hline
\end{tabular}
\end{table}
Ο αριθμός νημάτων (8) που υποστηρίζει ο επεξεργαστής i7-6500 οφείλεται στην
υποστήριξη της τεχνολογίας \emph{Hyper-Threading}\footnote{\url{http://www.intel.com/content/www/us/en/architecture-and-technology/hyper-threading/hyper-threading-technology.html}}.
Σημαντικό επίσης να αναφέρουμε ότι η τιμή της κατανάλωσης ισχύος της
πλατφόρμας Jetson TK1, αντιστοιχεί στην κατάσταση μέγιστης λειτουργίας των μονάδων
CPU και GPU.

\section{Πρώτη Φάση Πειραμάτων}
\label{sec:experiments_phase1}

TODO!!

\section{Δεύτερη Φάση Πειραμάτων}
\label{sec:experiments_phase2}

TODO!!

