{\fontfamily{cmr}\selectfont

\phantomsection
\addcontentsline{toc}{section}{Abstract}


\begin{center}
  \centering
  \textbf{\Large{Title}}
  \vspace{0.5cm}

  %\textbf{\large{Simultaneous Localization \& Mapping Combining \\Particle Filters, Critical Rays Scan Match \& Topological Information}}
  \textbf{\large{CNN Applications Towards Object Detection on Jetson TK1}}

  \vspace{1cm}

  \centering
  \textbf{Abstract}
\end{center}

Machine Learning has become crucial in recent times due to the evolution of
Artificial Neural Networks (ANN). Inspired by how brain network functions,
these computational models outperform previous forms of Artificial Intelligence
technics in common machine learning tasks.
Convolutional Neural Network (CNN) is one of the most fascinating forms of the ANN architecture,
used to solve pattern recognition tasks in Computer Vision (CV).
CNN is a form of representation learning algorithm, which means that no
hand-written features are required, enhancing machines with the ability
to extract proper features on their own, given a specific set of data.

Embedded systems have limited computational power compared to
desktop and server systems. On the other hand, embedded systems are been
designed for applications where low power consumption is a requirement,
for example in robotics. A robot must be provided with enough computational
power to perform basic tasks, like navigation, \\ localization, mapping
and object identification without consuming much power to exhaust the system.

The present Diploma Thesis focuses on implementions of CNN models for solving
object recognition and/or localization tasks, on the Jetson TK1 embedded system.
Several CNN models are been investigated like AlexNet, VGG16 and YOLO.
For each CNN model, the required memory (RAM), execution time and power consumption
rates are been measured in order to evaluate their efficiency. These rates are
been compared between those collected from executing on the Jetson TK1 platform
and a PC with an Intel-i7-6700 processor.

In addition, a set of software tools used for maximizing
performance per power consumption (performance/watt) ratio is presented,
along with proper configuration and
setup procedures for the Nvidia Jetson TK1 embedded system.

According to the outcome, selection of proper software for the implemention and
deployment of CNN models is critical. With proper software and a
series of optimization steps, the Jetson TK1 platform is able to
deliver solutions to general object detection tasks using CNNs, thus it
is considered to be appropriate for use in robotic applications.


\begin{flushright}
  \vspace{2cm}
  Konstantinos Panayiotou
  \\
  Electrical \& Computer Engineering Department,
  \\
  Aristotle University of Thessaloniki, Greece
  \\
  September 2016
\end{flushright}

}
