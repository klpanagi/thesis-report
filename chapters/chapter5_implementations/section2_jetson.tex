\section{Ρυθμίσεις και Βελτιστοποίησεις εργαλείων λογισμικού στο Jetson TK1}

Στο \autoref{sec:jetson_tk1} είχαμε αναφέρει ότι η πλατφόρμα τρέχει
λειτουργικό \emph{Linux4Tegra} το οποιό είναι ουσιαστικά διανομή Ubuntu 14.04
με εγκατεστημένους τους απαραίτητους drivers για το συγκεκριμένο σύστημα.

Το πρώτο βήμα ήταν να απεγκαταστήσουμε από το λειτουργικό το
γραφικό περιβάλλον, για να "κλέψουμε" μνήμη. Το Jetson TK1 συνδέθηκε
μέσω θύρας ethernet στο δίκτυο ενός σταθερού υπολογιστή και έτσι ο χειρισμός
του έγινε μέσω καναλιού ssh\footnote{SSH: Secure Shell \href{https://tools.ietf.org/html/rfc4254}{https://tools.ietf.org/html/rfc4254}}.

Στην συνέχεια εγκαταστήσαμε τις βιβλιοθήκες CUDA και cuDNN για να εκμεταλλευτούμε
την υπολογιστική ισχύ της μονάδας GPU.

Ένα μεγάλο μέρος της συνεισφοράς της παρούσας διπλωματική εργασίας
ήταν να γίνουν βελτιστοποιήσεις σε επίπεδο λογισμικού των εργαλείων που χρησιμοποιήθηκαν
για το συγκεκριμένο μηχάνημα.

Ένα από τα προβλήματα που συναντήσαμε ήταν το μέγεθος της μνήμης και το γεγονός
ότι η μνήμη είναι κοινή (shared memory) μεταξύ των μονάδων CPU και GPU.

%προτού ξεκινήσουμε τις υλοποίησεις χρειάστηκε να εγκαταστήσουμε
%τα περισσότερα εργαλεία κατευθείαν πάνω στο μηχάνημα (build from source), για να αποφύγουμε
%την εγκατάστασή pre-compiled λογισμικού. Ο λόγος, όπως και θα φανεί στο \autoref{chapter:experiments},
%είναι ότι η απόδοση των συγκεκριμένων εργαλείων εξαρτάται τόσο από τις
%βιβλιοθήκες που θα χρησιμοποιηθούν για πράξεις γραμμικής άλγεβρας,
%όσο και από τις βελτιστοποιήσεις που γίνονται από τον μεταγλωττιστή κατά την
%διάρκεια της μεταγλώττισης του εκάστοτε πηγαίου κώδικά.


\label{sec:implementations_jetson}

