\section{Μοντέλα CNN}
\label{sec:cnn_impl}

Όπως αναφέραμε \autoref{sec:dnn_sw} το βασικό εργαλείο λογισμικού που
χρησιμοποιήσαμε για την ανάπτυξη των μοντέλων CNN είναι το \emph{Keras}.

Η βιβλιοθήκη Keras προσφέρει τις υλοποιήσεις όλων των επιπέδων που
απαιτούνται για για την ανάπτυξη ενός CNN
\footnote{Πλήρες περιγραφή της λίστας των διαθέσιμων επιπέδων: \href{https://keras.io/layers/core/}{https://keras.io/layers/core/}}.
Πιο συγκεκριμένα, τα επίπεδα που χρησιμοποιήθηκαν, καθώς και οι βασικές
παράμετροι τους περιγράφονται πιο κάτω:
\begin{itemize}
  \item{InputLayer: Επίπεδο Εισόδου}
    \begin{itemize}
      \item{Διαστάσεις του όγκου εισόδου (tensor shape)}
    \end{itemize}
  \item{Convolution2D: Επίπεδο Συνέλιξης}
    \begin{itemize}
      \item{Μορφολογία της εισόδου}
      \item{Αριθμός των φίλτρων συνέλιξης}
      \item{Διαστάσεις των φίλτρων συνέλιξης}
      \item{Συνάρτηση ενεργοποίησης}
    \end{itemize}
  \item{MaxPooing2D: Επίπεδο Υποδειγματοληψίας}
    \begin{itemize}
      \item{Διαστάσεις πλαισίου}
      \item{Βήμα μετατόπισης}
    \end{itemize}
  \item{Activation: Επίπεδο ενεργοποίησης. Εφαρμόζει συνάρτηση ενεργοποίησης στον
    όγκο εξόδου του προηγούμενου επιπέδου}
  \item{Dropout: Επίπεδο πρόληψης υπέρ-προσαρμογής \cite{lecun2015deep}}
  \item{Dense: Πλήρες συνδεδεμένο επίπεδο}
    \begin{itemize}
      \item{Διαστάσεις του όγκου εισόδου (Προαιρετικό)}
      \item{Διαστάσεις του όγκου εξόδου}
      \item{Συνάρτηση ενεργοποίησης}
    \end{itemize}
  \item{Flatten: Μετασχηματίζει τον όγκου εισόδου σε επίπεδη αναπαράσταση
    (π.χ. για όγκο εισόδου $64 \times 32 \times 32$ η έξοδος θα είναι επίπεδη με $65536$ νευρώνες}
  \item{BatchNormalization: Εφαρμόζει μετασχηματισμό για να κρατήσει την μέση τιμή
      και την τυπική απόκλιση των ενεργοποιήσεων του προηγούμενου επιπέδου στις τιμές 0 και 1 αντίστοιχα}
\end{itemize}

Σημαντική παρατήρηση είναι η επιστήμη της βαθιάς μηχανικής μάθησης
βρίσκεται σε πρώιμο στάδιο, με αποτέλεσμα να μην υπάρχουν συγκεντρωμένες
οι υλοποιήσεις των διαφόρων επιπέδων και γενικότερα των μοντέλων σύγχρονων ΑNN.

Περαιτέρω, η επιλογή των μοντέλων CNN για ανάπτυξη στηρίχθηκε στην ύπαρξη και
προ-εκπαιδευμένων βαρών για τα αντίστοιχα CNN στο διαδίκτυο για 2 λόγους:
\begin{itemize}
  \item{Η διαδικασία εκπαίδευσης είναι χρονοβόρα διαδικασία και προϋποθέτει
    τη χρησιμοποίηση μίας ή περισσοτέρων ισχυρών μονάδων GPU (NVIDIA Titam X GPU)}
  \item{Η εκπαίδευση νευρωνικών δικτύων ξεφεύγει από τα πλαίσια της παρούσας
    διπλωματικής εργασίας}
\end{itemize}

\subsection{AlexNet}

\subsection{VGG16}

\subsection{VGG19}

\subsection{GoogleNet aka Inception-V1}
