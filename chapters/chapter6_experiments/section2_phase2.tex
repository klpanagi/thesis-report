\section{Δεύτερη Φάση Πειραμάτων}
\label{sec:experiments_phase2}

Η μεταγλώττιση των δικτύων AlexNet και VGG16 απαιτεί περισσότερη
μνήμη από αυτή που διαθέτει το ενσωματωμένο σύστημα Jetson TK1 (1888MB).
Το πρόβλημα αυτό το αντιμετωπίστηκε προσθέτοντας 1GB μνήμη
\emph{Swap}\footnote{Δημιουργία αρχείου μνήμης swap: \url{http://www.jetsonhacks.com/2014/10/04/creating-swapfile-ubuntu-nvidia-jetson-tk1/}}.

Ο λόγος της απόδοσης ανά μονάδα ισχύος (performance per watt) υπολογίστηκε
με βάση τον αριθμό τον εικόνων που μπορεί να επεξεργαστεί το εκάστοτε
νευρωνικό δίκτυο ανά δευτερόλεπτο (frames per second).

Οι εκδόσεις των εργαλείων λογισμικού που χρησιμοποιήθηκαν
κατά την διάρκεια των πειραμάτων για τον επεξεργαστή Intel-i7-6700 είναι:
\begin{itemize}
  \item{Keras: v1.1.0}
  \item{Theano: v0.9.0-dev3}
  \item{numpy: v1.11.0}
  \item{scipy: v0.19.0-dev0}
\end{itemize}

%%----------------------------------------------------------------------------

\subsection{AlexNet}

Οι παράμετροι των πειραμάτων είναι οι εξής:
\begin{itemize}
  \item{Αριθμός επαναλήψεων: 1000}
  \item{Υπολογιστική μονάδα:}
    \begin{itemize}
      \item{CPU: Αριθμός Νημάτων (1, 2, 4, 8)}
      \item{GPU: Με και χωρίς εκ των προτέρων δέσμευση μνήμης}
    \end{itemize}
\end{itemize}

\begin{figure}[H]
  \centering
  \includegraphics[width=0.8\textwidth]{./images/chapter6/benchmark_alexnet_jetson.png}
  \caption[Χρόνoι εκτέλεσης για το δίκτυο AlexNet στο Jetson TK1]{Χρόνoι εκτέλεσης για το δίκτυο AlexNet στο Jetson TK1}
  \label{fig:alexnet_results_jetson}
\end{figure}

Στον \autoref{tab:alexnet_jetson} παρουσιάζονται τα συγκριτικά αποτελέσματα της
εκτέλεσης τόσο στην μονάδα CPU (με μεταβλητό αριθμό νημάτων), όσο και στην
GPU (με και χωρίς εκ των προτέρων δέσμευση μνήμης).

\begin{table}[H]
  \begin{center}
    \caption{Μετρήσεις πειραμάτων για το δίκτυο AlexNet στο Jetson TK1}
    \label{tab:alexnet_jetson}
    \small
    \begin{tabular}[center]{ | c | c | c | c | c | c | }
      \hline
      \rowcolor{Gray}
      Μονάδα & \# νημάτων & cnmem & Χρόνος εκτέλεσης (sec) & Κατ. Ισχύος (Watts) & Perf/Watt \\
      \hline
      CPU & 1 & N/A & 0.62 & 6.318 & 0.254\\
      CPU & 2 & N/A & 0.4845 & 8.019 & 0.2574\\
      CPU & 4 & N/A & 0.4634 & 10.692 & 0.202\\
      CPU & 8 & N/A & 0.4646 & 10.692 & 0.2013\\
      GPU & N/A & None & 0.1049 & 8.5 & 1.1215\\
      GPU & N/A & 700MB & 0.1 & 8.5 & 1.1764\\
      \hline
    \end{tabular}
  \end{center}
\end{table}


%%----------------------------------------------------------------------------

\subsection{VGG16}

Οι παράμετροι των πειραμάτων είναι οι εξής:
\begin{itemize}
  \item{Αριθμός επαναλήψεων: 100}
  \item{Υπολογιστική μονάδα:}
    \begin{itemize}
      \item{CPU: Αριθμός Νημάτων (1, 2, 4, 8)}
      \item{GPU: Με και χωρίς εκ των προτέρων δέσμευση μνήμης (cnmem)}
    \end{itemize}
\end{itemize}

Στον παρακάτω \autoref{tab:vgg16_jetson} βρίσκονται τα συγκριτικά αποτελέσματα της
εκτέλεσης τόσο στην μονάδα CPU (με μεταβλητό αριθμό νημάτων), όσο και στην
GPU (με και χωρίς εκ των προτέρων δέσμευση μνήμης).

\begin{table}[H]
  \begin{center}
    \caption{Μετρήσεις πειραμάτων για το δίκτυο VGG16 στο Jetson TK1}
    \label{tab:vgg16_jetson}
    \small
    \begin{tabular}[center]{ | c | c | c | c | c | c | }
      \hline
      \rowcolor{Gray}
      Μονάδα & \# νημάτων & cnmem & Χρόνος εκτέλεσης (sec) & Κατ. Ισχύος (Watts) & Perf/Watt \\
      \hline
      CPU & 1 & N/A & 9.431 & 6.318 & 0.0167\\
      CPU & 2 & N/A & 5.4476 &  8.748 & 0.021\\
      CPU & 4 & N/A & 3.6224 & 13.122 & 0.021\\
      CPU & 8 & N/A & 3.6308 & 13.122 & 0.021\\
      GPU & N/A & None & 0.6983 & 11.907 & 0.1207\\
      GPU & N/A & 720MB & 0.5761 & 11.907 & 0.1457\\
      \hline
    \end{tabular}
  \end{center}
\end{table}

Στο \autoref{fig:vgg16_results_jetson} που ακολουθεί, παρουσιάζεται το διάγραμμα
των χρόνων εκτέλεσης στην μονάδα CPU με χρήση τεσσάρων νημάτων και στην μονάδα
GPU με εκ των προτέρων δέσμευση 720MB μνήμης.

\begin{figure}[H]
  \centering
  \includegraphics[width=0.8\textwidth]{./images/chapter6/benchmark_vgg16_jetson.png}
  \caption[Χρόνoι εκτέλεσης για το δίκτυο VGG16 στο Jetson TK1]{Χρόνoι εκτέλεσης για το δίκτυο VGG16 στο Jetson TK1}
  \label{fig:vgg16_results_jetson}
\end{figure}

Παρατηρούμε ότι η εκτέλεση στην μονάδα GPU είναι 5.187 φορές πιο
γρήγορη (χωρίς εκ των προτέρων δέσμευση μνήμης).
Επίσης, με εκ των προτέρων δέσμευση μνήμης για την μονάδα GPU, ο χρόνος εκτέλεσης μειώνεται ακόμη
περισσότερο (περίπου 120ms).



%%----------------------------------------------------------------------------

\subsection{Tiny-YOLO}

Οι παράμετροι των πειραμάτων είναι οι εξής:
\begin{itemize}
  \item{Αριθμός επαναλήψεων: 1000}
  \item{Υπολογιστική μονάδα:}
    \begin{itemize}
      \item{CPU: Αριθμός Νημάτων (1, 2, 4, 8)}
      \item{GPU: Με και χωρίς εκ των προτέρων δέσμευση μνήμης (cnmem)}
    \end{itemize}
\end{itemize}

\begin{table}[H]
  \begin{center}
    \caption{Μετρήσεις πειραμάτων για το δίκτυο Tiny-YOLO στο Jetson TK1}
    \label{tab:yolo_jetson}
    \small
    \begin{tabular}[center]{ | c | c | c | c | c | c | }
      \hline
      \rowcolor{Gray}
      Μονάδα & \# νημάτων & cnmem & Χρόνος εκτέλεσης (sec) & Κατ. Ισχύος (Watts) & Perf/Watt \\
      \hline
      CPU & 1 & N/A & 1.5902 & 6.318 & 0.1\\
      CPU & 2 & N/A & 1.0276 & 8.504 & 0.1144\\
      CPU & 4 & N/A & 0.8137 & 11.9 & 0.103\\
      CPU & 8 & N/A & 0.8126 & 11.9 & 0.1034\\
      GPU & N/A & None & 0.5543 & 8 & 0.2255\\
      GPU & N/A & 400MB & 0.3872 & 8 & 0.323\\
      \hline
    \end{tabular}
  \end{center}
\end{table}

\begin{figure}[H]
  \centering
  \includegraphics[width=0.8\textwidth]{./images/chapter6/benchmark_yolotiny_jetson.png}
  \caption[Χρόνoι εκτέλεσης για το δίκτυο Tiny-YOLO στο Jetson TK1]{Χρόνοι εκτέλεσης για το δίκτυο Tiny-YOLO στο Jetson TK1}
  \label{fig:yolotiny_results_jetson}
\end{figure}

Ο χρόνος εκτέλεσης της αντίστοιχης υλοποίησης της ερευνητικής ομάδας που
σχεδίασε το δίκτυο Tiny-YOLO μετρήθηκε, για την μονάδα CPU του Tegra K1,
στα 3.475 δευτερόλεπτα (τέσσερις φορές πιο αργό από την υλοποίηση
μας).
Η μέτρηση του χρόνου εκτέλεση της αντίστοιχης
υλοποίησης στην μονάδα GPU ήταν αδύνατη. Η εκτέλεση στην μονάδα GPU
απαιτούσε περισσότερη μνήμη από αυτή που διαθέτει η πλατφόρμα Jetson TK1 (2GB) και έτσι
το εκτελέσιμο τερματίζει με σφάλμα \emph{CUDA Error: out of memory}.
Ο λόγος που αυτό συμβαίνει μόνο στην περίπτωση εκτέλεσης στην μονάδα GPU
δεν είναι εμφανές. Πιθανόν να οφείλεται σε σφάλμα τύπου διαρροής μνήμης (memory leak).

