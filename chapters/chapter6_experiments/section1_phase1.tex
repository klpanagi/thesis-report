\section{Πρώτη Φάση Πειραμάτων}
\label{sec:experiments_phase1}

\subsection{AlexNet}

Οι παράμετροι των πειραμάτων είναι οι εξής:
\begin{itemize}
  \item{Αριθμός επαναλήψεων: 1000}
  \item{Αριθμός Νημάτων: 1, 2, 4, 8}
\end{itemize}

Στον πιο κάτω πίνακα παρουσιάζονται τα αποτελέσματα της μέσης τιμής
του χρόνου εκτέλεσης, ενώ στο \autoref{fig:alexnet_results_i7} φαίνεται
το διάγραμμα του χρόνου εκτέλεσης σε κάθε επανάληψη.

\begin{center}
  \small
  \begin{tabular}{ | c | c | c | c | c | }
    \hline
    \rowcolor{Gray}
    \# νημάτων & 1 & 2 & 4 & 8 \\
    \hline
    Χρόνος εκτέλεσης (sec) & $0.117$ & $0.095$ & $0.095$ & $0.127$ \\
    \hline
  \end{tabular}
\end{center}


%Η υλοποίηση του δικτύου AlexNet δεσμεύει \emph{692MB} μνήμη RAM.
Η μέγιστη τιμή μνήμης RAM που δεσμεύει το δίκτυο AlexNet μετρήθηκε στα \emph{692MB}.

\begin{figure}[!ht]
  \centering
  \includegraphics[width=0.8\textwidth]{./images/chapter6/benchmark_alexnet_i7.png}
  \caption[Χρόνoι εκτέλεσης για το δίκτυο AlexNet σε επεξεργαστή i7]{Χρόνοι εκτέλεσης για το δίκτυο AlexNet σε επεξεργαστή i7}
  \label{fig:alexnet_results_i7}
\end{figure}



%%----------------------------------------------------------------------------

\subsection{VGG16}

Οι παράμετροι των πειραμάτων είναι οι εξής:
\begin{itemize}
  \item{Αριθμός επαναλήψεων: 100}
  \item{Αριθμός Νημάτων: 1, 2, 4, 8}
\end{itemize}

Η μέσες τιμές του χρόνου εκτέλεσης των πειραμάτων παρουσιάζονται στον πιο κάτω πίνακα:

\begin{center}
  \small
  \begin{tabular}{ | c | c | c | c | c | }
    \hline
    \rowcolor{Gray}
    \# νημάτων & 1 & 2 & 4 & 8 \\
    \hline
    Χρόνος εκτέλεσης (sec) & $0.7709$ & $0.5577$ & $0.4734$ & $0.6555$ \\
    \hline
  \end{tabular}
\end{center}

Η απαιτούμενη τιμή μνήμης RAM μετρήθηκε στα \emph{710MB}.

\begin{figure}[!ht]
  \centering
  \includegraphics[width=0.8\textwidth]{./images/chapter6/benchmark_vgg16_i7.png}
  \caption[Χρόνoι εκτέλεσης δικτύου VGG16 σε επεξεργαστή i7]{Χρόνοι εκτέλεσης δικτύου VGG16 σε επεξεργαστή i7}
  \label{fig:vgg16_results_i7}
\end{figure}



%%----------------------------------------------------------------------------

\subsection{Tiny-Yolo}

Οι παράμετροι των πειραμάτων είναι οι εξής:
\begin{itemize}
  \item{Αριθμός επαναλήψεων: 1000}
  \item{Αριθμός Νημάτων: 1, 2, 4, 8}
\end{itemize}

Η αντίστοιχη μέση τιμή των χρόνων εκτέλεσης των πειραμάτων παρουσιάζονται στον πιο κάτω πίνακα:

\begin{center}
  \small
  \begin{tabular}{ | c | c | c | c | c | }
    \hline
    \rowcolor{Gray}
    \# νημάτων & 1 & 2 & 4 & 8 \\
    \hline
    Χρόνος εκτέλεσης (sec) & $0.18709$ & $0.1609$ & $0.1516$ & $0.2309$ \\
    \hline
  \end{tabular}
\end{center}

\begin{figure}[!ht]
  \centering
  \includegraphics[width=0.8\textwidth]{./images/chapter6/benchmark_yolotiny_i7.png}
  \caption[Χρόνoι εκτέλεσης δικτύου Tiny-YOLO σε επεξεργαστή i7]{Χρόνοι εκτέλεσης δικτύου Tiny-YOLO σε επεξεργαστή i7}
  \label{fig:yolotiny_results_i7}
\end{figure}

Η μνήμη (μέγιστη) που απαιτείται κατά την διαδικασία
προς-τα-εμπρός εκτέλεσης μετρήθηκε στα \emph{379MB}.

Σημαντικό να αναφερθεί ότι ο χρόνος εκτέλεσης της αντίστοιχης υλοποίησης της ερευνητικής ομάδας
που σχεδίασε το δίκτυο Tiny-YOLO, μετρήθηκε στα \emph{1.096 δευτερόλεπτα}.

