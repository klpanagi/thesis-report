\chapter{Συμπεράσματα}
\label{chapter:conclusions}

Στο παρόν κεφάλαιο παρουσιάζονται συνοπτικά οι υλοποιήσεις των μοντέλων CNN
και τα συμπεράσματα στα οποία καταλήξαμε με βάση τα αποτελέσματα των πειραμάτων.
Γίνεται σύγκριση της απόδοσης των μοντέλων αυτών στα δύο συστήματα
(PC με επεξεργαστή Intel i7-6700 και Jetson TK1), έχοντας για
παραμέτρους αξιολόγησης κυρίως τον χρόνο εκτέλεσης αλλά και την κατανάλωση ισχύος.
Στην συνέχεια αναφέρονται τα εμπόδια και προβλήματα που παρουσιάστηκαν
κατά την διάρκεια των υλοποιήσεων και των πειραμάτων.

\section{Γενικά Συμπεράσματα}

Στα πλαίσια της παρούσας διπλωματικής εργασίας υλοποιήθηκε το δίκτυο
Tiny-YOLO χρησιμοποιώντας την βιβλιοθήκη Keras.
Τo δίκτυο AlexNet υλοποιήθηκε με βάση την αντίστοιχη υλοποίηση
στο Caffe \footnote{Υλοποίηση του δικτύου AlexNet στο Caffe: \url{https://github.com/BVLC/caffe/tree/master/models/bvlc_alexnet}}.
ενώ το μοντέλο του δικτύου VGG16 υπάρχει υλοποιημένο στα παραδείγματα του Keras.

Με βάση τα αποτελέσματα των πειραμάτων (\autoref{chapter:experiments})
που αφορούν στους χρόνους εκτέλεσης και κατανάλωσης ισχύος παρατηρήθηκαν τα
εξής:
\begin{itemize}
  \item{Οι καλύτεροι χρόνοι εκτέλεσης στον επεξεργαστή Intel-i7-6700 (και για τα 3 μοντέλα CNN)
      λήφθηκαν με χρήση τεσσάρων (4) νημάτων, παρόλο που ο συγκεκριμένος επεξεργαστής υποστηρίζει
      (ψευδο)παράλληλη εκτέλεση 8 νημάτων. Αυτό οφείλεται στο γεγονός ότι
      η τεχνολογία Hyper-Threading που υποστηρίζει ο συγκεκριμένος επεξεργαστής
      παρόλο που επιτρέπει την δρομολόγηση 2
      νημάτων σε κάθε πυρήνα του επεξεργαστή, η εναλλαγή (ανά δύο) των
      νημάτων στον κάθε πυρήνα (context switch\footnote{\url{http://wiki.osdev.org/Context_Switching}})
      είναι γεγονός που προσδίδει καθυστέρηση στην εκτέλεση.}

  \item{Παρατηρήθηκε καλύτερος χρόνος εκτέλεσης χρησιμοποιώντας 4 νήματα
      για τους υπολογισμούς πράξεων γραμμικής άλγεβρας. Με χρήση
      8 νημάτων ο χρόνος εκτέλεσης αυξήθηκε κατά 32ms, γεγονός που οφείλεται
    στο \emph{context-switch}...}
  \item{Ο χρόνος εκτέλεσης στις μονάδες CPU, παρουσιάζει διακυμάνσεις,
      ιδιαίτερα με χρήση αριθμού νημάτων μικρότερου από τον αριθμό
      των διαθέσιμων πυρήνων του εκάστοτε επεξεργαστή.
    Το παραπάνω αποτελεί λογικό αποτέλεσμα και οφείλεται στο context-switch}
  \item{Με χρήση 4 νημάτων το συγκε}
\end{itemize}


% ----------------------------------------------------------------------------

\section{Προβλήματα}

ΒΛΑΒΛΑΒΛ
