\chapter{Εισαγωγή}
\label{chapter:intro}

Ο όρος \emph{Τεχνητή Νοημοσύνη} αναφέρεται στην ικανότητα των υπολογιστικών
συστημάτων να μιμούνται στοιχεία της ανθρώπινης συμπεριφοράς.
Η επιθυμία των ανθρώπων να κατασκευάσουν "έξυπνες" μηχανές, καταγράφεται από
την εποχή της αρχαίας Ελλάδας. Μυθικές μορφές όπως οι Πυγμαλίωνας, Δαίδαλος και
Ήφαιστος μπορούν να θεωρηθούν ως θρυλικοί εφευρέτες και δημιουργοί νοούντων
μηχανών όπως η Γαλάτεια, ο Τάλος και η Πανδώρα.

Ένας πιο ολοκληρωμένος ορισμός της Τεχνητής Νοημοσύνης είναι:
\begin{displayquote}
\emph{
  Ο κλάδος/τομέας της επιστήμης της πληροφορικής, που ασχολείται
  με την σχεδίαση και κατασκευή ευφυών συστημάτων, δηλαδή συστημάτων που
  διαθέτουν χαρακτηριστικά που σχετίζονται με την ανθρώπινη νοημοσύνη και συμπεριφορά.
}
\end{displayquote}
Με την εμφάνιση των πρώτων ηλεκτρονικών και (επανα)προγραμματιζόμενων υπολογιστικών συστημάτων,
οι άνθρωποι ξεκίνησαν να σκέφτονται τρόπους για να κατασκευάσουν "έξυπνες" μηχανές.
H ραγδαία εξέλιξη στον κλάδο της επιστήμης της πληροφορικής τις τελευταίες
δεκαετίες, έφερε και την εξέλιξη στην επιστήμη της Τεχνητής Νοημοσύνης.
Το 1997, η IBM κατασκεύασε ένα υπολογιστικό σύστημα το οποίο μπορούσε να
παίξει σκάκι (Deep Blue) \cite{campbell2002deep}, το οποίο κέρδισε τον παγκόσμιο %ref does not exist
τότε πρωταθλητή στο σκάκι, Garry Kasparov. Το σκάκι έχει εξήντα τέσσερις θέσεις
και τριάντα δύο πιόνια που μπορούν να κινούνται με συγκεκριμένο τρόπο. H μηχανή
Deep Blue, μπορούσε να εκτιμήσει και να αξιολογήσει διακόσια εκατομμύρια
πιθανές καταστάσεις της σκακιέρας. Ωστόσο, πρέπει να σημειώσουμε ότι η επίλυση του
προβλήματος του σκακιού, παρ' ότι είναι ένα πρόβλημα το οποίο μπορεί να περιγραφεί
πλήρως μέσα από μια λίστα με κανόνες, δεν είναι δυνατό να λυθεί αναλυτικά εξαιτίας
του τεράστιου αριθμού των δυνατών κινήσεων. 
%\begin{figure}[!ht]
  %\centering
  %\includegraphics[width=0.7\textwidth]{./images/chapter3/building_a_brain.jpg}
  %% \caption[Τεχνητή Νοημοσύνη]{Τεχνητή Νοημοσύνη}
  %\label{fig:AI_1}
%\end{figure}

Η "δύναμη" της τεχνητής νοημοσύνης υπερβαίνει τα όρια της φαντασίας.
Τα ρομποτικά συστήματα ήδη έχουν περάσει φάση δοκιμής, σε διάφορες χώρες ανά τον
κόσμο, όσον αφορά την ένταξή τους στην καθημερινότητα του ανθρώπου.

Η δυνατότητα ενός "ευφυούς" ρομποτικού συστήματος να αντιλαμβάνεται
το περιβάλλον του, είναι απαραίτητη ικανότητα που πρέπει να διαθέτει.
Ένας ρομποτικός πράκτορας, σε πληθώρα εφαρμογών, πρέπει να διαθέτει τόσο την
ικανότητα να αναγνωρίζει, μέσω εικόνων λήψης από κάμερα, διάφορα αντικείμενα,
όσο και την ικανότητα να εντοπίζει επακριβώς την θέση του
εκάστοτε αντικειμένου στον χώρο. Δεν θα είχε νόημα για ένα ρομπότ, να έχει άποψη
για την παρουσία αντικειμένων γύρω του, αν δεν έχει και την ικανότητα να γνωρίζει
και την θέση των αντικειμένων αυτών. % not exactly but...

Ένας σημαντικός παράγοντας της ραγδαίας εξέλιξης της επιστήμης της ρομποτικής, αλλά
και της τεχνητής νοημοσύνης γενικότερα, είναι η εμφάνιση και εξέλιξη του κλάδου
της \emph{Βαθιάς Μηχανικής Μάθησης - Deep learning}.

Η χρήση τεχνικών βαθιάς μάθησης στην επίλυση προβλημάτων \emph{Μηχανικής Όρασης},
έχει φέρει την συγκεκριμένη τεχνολογία σε θέση να μπορεί να αντιμετωπίσει
περίπλοκα προβλήματα τα οποία μέχρι και πριν λίγα χρόνια θεωρείτο ακατόρθωτο να λυθούν.
Ένα στοχευμένο παράδειγμα είναι τα αυτόνομα αμάξια, τα οποία σήμερα είναι σε
φάση δοκιμής.

Σήμερα, το γενικότερο πρόβλημα της αναγνώρισης και εντοπισμού αντικειμένων σε εικόνες
λύνεται με την χρήση Νευρωνικών Δικτύων Συνέλιξης (Convolutional Neural Networks - CNN).
Το γεγονός ότι τα CNN έχουν την δυνατότητα να κατηγοριοποιήσουν αντικείμενα
σε προβλήματα όπου οι κλάσεις αντικειμένων κυμαίνονται από δεκάδες έως και χιλιάδες,
τα καθιστά ικανά να χρησιμοποιηθούν σε πραγματικού χρόνου εφαρμογές
και ιδιαίτερα στην επιστήμη της ρομποτικής.

\section{Περιγραφή του Προβλήματος}
\label{section:problem_description}

Παρόλο που τα CΝΝ είναι ικανά να δώσουν λύσεις με μεγάλη ακρίβεια, απαιτούν
μεγάλο όγκο επεξεργαστικής ισχύ, τόσο για την εκπαίδευσή τους, όσο και για την
εκτέλεση ενός πειράματος, όταν το πρόβλημα για το οποίο έχουν
σχεδιαστεί να δώσουν λύση είναι περίπλοκο. Η απαίτηση αυτή προέρχεται από το βάθος
των μοντέλων CNN για αναγνώριση και εντοπισμό αντικειμένων σε εικόνες.
Για παράδειγμα, ένα από τα πρώτα μοντέλα CNN το οποίο σχεδιάστηκε για την
αναγνώριση αντικειμένων σε εικόνες, αποτελείτε από 16 επίπεδα (AlexNet)
και έχει εξήντα εκατομμύρια (60,000,000) παραμέτρους και
εξακόσιες-πενήντα χιλιάδες (650,000) νευρώνες από τους οποίους οι περισσότεροι
εκτελούν πράξεις συνέλιξης. Ο Alex Krizhevsky απέδειξε το 2012 ότι η χρήση
σύγρονων GPU για την εκτέλεση πράξεων συνέλιξης, φέρει σαν αποτέλεσμα την
εκαπάίδευση μοντέλων CNN σε χρόνους εώς και δύο τάξεις μεγέθους πιο κάτω σε σχέση
με έναν ισχυρό επεξεργαστή \cite{NIPS2012_4824}. Ο χρόνος εκτέλεσης ενός πειράματος του Δικτύου AlexNet
έχει μετρηθεί στα 7.39 δευτερόλεπτα σε οκτα-πύρηνο επεξεργαστή Haskwell @2.9Ghz
και στα 0.71 δευτερόλεπτα σε μονάδα GPU, Nvidia K520 \cite{abuzaidoptimizing}.

Είναι ιδιαίτερα σημαντικό, ένα ρομπότ να μπορεί να αντιλαμβάνεται το περιβάλλον
του; να μπορεί να αναγνωρίζει ανθρώπους, ζώα, αντικείμενα γενικότερα. Ωστόσο,
θέλουμε τα ρομπότ να είναι και όσο πιο "ελκυστικά" γίνεται στον άνθρωπο ή/και μικρότερα.
Αυτό, φέρει σαν αποτέλεσμα να μην μπορούμε να τοποθετήσουμε ογκώδη, άρα με μεγάλη
υπολογιστική ισχύ, υπολογιστικά συστήματα στο σώμα των ρομποτικών συστημάτων.

Παρόλο που σήμερα έχουν σχεδιαστεί μοντέλα CNN, τα οποία έχουν την δυνατότητα να
αναγνωρίσουν και να εντοπίσουν αντικείμενα από χιλιάδες, αν όχι και περισσότερες,
κλάσεις, ο χρόνος που απαιτείται για να κατηγοριοποιήσει αντικείμενα σε μία εικόνα
είναι αρκετά μεγάλος, της τάξης των μερικών δευτερολέπτων σε σύνχρονους υπολογιστές.
Αυτό κάνει την χρήση CNN σε εφαρμογές πραγματικού χρόνου, όπως για παράδειγμα
στην ρομποτική, ακατάλληλη.
Ωστόσο, η επιστημονική κοινότητα σήμερα προσπαθεί να δώσει λύσεις στο συγκεκριμένο
πρόβλημα εστιάζοντας το ενδιαφέρον στην εξέλιξη των ενσωματωμένων συστημάτων
και την σχεδίαση γρήγορων λογισμικών για υλοποιήσεις μοντέλων DNN τα οποία
εκμεταλλεύονται κυρίως την υπολογιστική ισχύ των μονάδων GPU, αλλά και άλλων
πολυπύρηνων επεξεργαστικών μονάδων.

\section{Σκοπός - Συνεισφορά της Διπλωματικής Εργασίας}
\label{section:contribution}


\section{Διάρθρωση της Αναφοράς}
\label{section:layout}

Η διάρθρωση της παρούσας διπλωματικής εργασίας είναι η εξής:

\begin{itemize}
  \item{\textbf{Κεφάλαιο \ref{chapter:sota}:} Παρατίθεται η ανασκόπηση της ερευνητικής
      περιοχής αναφορικά με τα αντικείμενα στα οποία επιδιώκει να
      παρουσιάσει λύσεις η διπλωματική εργασία.
    }
  \item{\textbf{Κεφάλαιο \ref{chapter:theory}:} Περιγράφονται τα βασικά θεωρητικά στοιχεία
      στα οποία βασίστηκαν οι υλοποιήσεις.
    }
  \item{\textbf{Κεφάλαιο \ref{chapter:tools}:} Παρουσιάζονται οι
      διάφορες τεχνικές και τα εργαλεία που χρησιμοποιήθηκαν.
    }
  \item{\textbf{Κεφάλαιο \ref{chapter:implementations}:} Πλήρες περιγραφή των υλοποιήσεων.
    }
  \item{\textbf{Κεφάλαιο \ref{chapter:experiments}:} Παρουσιάζεται αναλυτικά η μεθοδολογία των
      πειραμάτων.
    }
  \item{\textbf{Κεφάλαιο \ref{chapter:conclusions}:} Παρουσιάζονται τα συμπεράσματα στα οποία
      καταλήξαμε.
    }
  \item{\textbf{Κεφάλαιο \ref{chapter:future_work}:} Στο τελευταία αυτό κεφάλαιο αναφέρονται τα
      προβλήματα που προέκυψαν και προτείνονται θέματα για μελλοντική
      μελέτη, αλλαγές και επεκτάσεις.
    }
\end{itemize}




